\documentclass[8pt, reqno, hidelinks]{amsart}
%\documentclass[8ptpt]{report}
\usepackage[margin=1.0cm]{geometry}
\usepackage[utf8]{inputenc}
\usepackage{multicol, wrapfig, blindtext, booktabs}
\usepackage{amsmath, amsthm, amsfonts, amssymb, mathrsfs, physics}
\usepackage{stackrel}
\usepackage{dsfont}
\usepackage{graphicx}
\usepackage{booktabs}
\usepackage{minted}
\usepackage{hyperref, bookmark}
\usepackage{cancel}
\usepackage{extarrows}
\usepackage{pagecolor}
\usepackage{tikz}
\usepackage{tikz-cd}
\usepackage{svg}
\usepackage{subcaption}
\usepackage{mathtools}
\usepackage{centernot}
\usepackage{xcolor}
\DeclarePairedDelimiter{\ceil}{\lceil}{\rceil}
\newenvironment{Figure}
  {\par\medskip\noindent\minipage{\linewidth}}
  {\endminipage\par\medskip}

% Standard Sets
\newcommand{\naturals}{\mathbb{N}}
\newcommand{\whole}{\mathbb{Z}}
\newcommand{\rationals}{\mathbb{Q}}
\newcommand{\reals}{\mathbb{R}}
\newcommand{\complex}{\mathbb{C}}
\newcommand{\field}{\mathbb{K}}
\newcommand{\cm}{\text{CM}}
\newcommand{\due}[1]{{^\star}#1}

\definecolor{blue}{HTML}{0044cc}
\definecolor{red}{HTML}{ff3333}
\definecolor{light-green}{HTML}{cdff99}
\definecolor{light-purple}{HTML}{ffcccc}
\definecolor{light-orange}{HTML}{ffdd99}



%\newcommand{\mychapter}[1]{\section*{\texorpdfstring{\colorbox{light-purple}{#1}}{#1}}}
\newcommand{\mychapter}[1]{\section*{\texorpdfstring{\fbox{#1}}{#1}}}
\newcommand{\mysection}[1]{\section*{\texorpdfstring{\colorbox{light-orange}{#1}}{#1}}}
\newcommand{\mydefinition}[1]{\subsection*{\textcolor{red}{#1}}}
\newcommand{\mytheorem}[1]{\subsection*{\textcolor{blue}{#1}}}
\newcommand{\myparagraph}[1]{\paragraph{\textbf{#1}}}
\newcommand{\myproof}[1]{\tiny{\textcolor{gray}{#1}}}



\begin{document}
\mychapter{Appunti}

\begin{multicols}{3}
  \mydefinition{Forze di natura elettrica}
  La forza elettrica \'e una formza con azione a distanza osservata inizialmente
  da esperienze comuni come l'attrazione di pezzetti di carta ad una bachetta di
  vetro strofinata con della pelle di gatto (vivo sperabilmente).

  \mydefinition{Ipotesi di Franklin (da revisionare)}
  Esistono 2 classi di materiali: positivi (come il vetro) e negativi
  (come la bachelite).

  \mydefinition{Modello atomico di Thomson}
  Il modello atomico di Thomson (1898) noto come modello a panettone prevede un
  corpo carico positivamente popolato da corpi minori di carica negativa
  all'interno. Il modello venne superato a breve da Rutherford (1909)
  attraverso le sue osservazioni dall'esperimento di scattering; esso spiega
  la deflessione di particelle cariche positivamente con un modello planetario
  dell'atomo che coinvolge un denso nucleo positivo a cui orbitano attorno le
  cariche negative.

  \mydefinition{Ordini di grandezza (distanze)}
  La grandezza di un atomo \'e nell'ordine di $10^{-10}m$ mentre il suo nucleo
  nell'ordine di $10^{-15}m$. A queste scale microscopiche si ricorre usualmente
  ad una scala pi\'u conveniente detta Angstrom $1\mathring{A} = 10^{-10}m$

  \mydefinition{La sostanza della materia}
  La massa del protone $m_p\approx 1.6725\cdot 10^{-27} kg$,
  la massa dell'elettrone $m_e \approx 9.1094 \cdot 10^{-31} kg$ e
  la massa del neutrone $m_n\approx 1.6748\cdot 10^{-27} kg$.
  La carica dell'elettrone  $e = 1.69 \cdot 10^{-19} C$ viene detta anche
  \textit{carica elementare} siccome ogni altro corpo carico deve essere un
  multiplo intero di questa.
  

  \mysection{Metodi di elettrizzazione}
  \mydefinition{Triboelettrico}
  Fenomeno elettrico che consiste nel trasferimento di cariche elettriche tra
  materiali neutri diversi (di cui almeno uno isolante) quando vengono
  \textbf{strofinati} tra di loro.

  \mydefinition{Induzione}
  Il fenomeno per cui la carica elettrica all'interno di un oggetto viene
  ri-distribuita a causa della presenza di un altro oggetto carico nelle
  \textbf{vicinanze}.

  \mydefinition{Conduzione}
  Fenomeno elettrico che consiste nel trasferimento di cariche elettriche tra
  un materiale carico ed uno neutro quando vengono messi a \textbf{contatto}
  fra di loro.

  \mydefinition{Legge di conservazione della carica}
  In un sistema isolato la carica complessiva si conserva.
  \begin{equation}
    \label{eq:conservazione-carica}
    \sum_i \frac{\Delta q_i(t)}{\Delta t} = 0
  \end{equation}

  \mydefinition{Legge di coulomb}
  \begin{equation}
    \label{eq:legge-coulomb}
    \vec{f}_{12} = k\frac{q_1q_2}{r^2}\hat{r}_{12}
  \end{equation}

  \mytheorem{Ordini di grandezza (forze)}
  \begin{equation}
    \label{eq:odg-forze}
    \begin{gathered}
      F_E = \frac{1}{4\pi\varepsilon_0}\frac{e^2}{r_0^2} \approx 10^{-7} N \\
      F_G = G\frac{m_pm_e}{r_0^2} \approx 10^{-47} N
    \end{gathered}
  \end{equation}

  \mydefinition{Campi}
  Un campo scalare \'e una funzione che associa ad ogni coppia di posizione ed
  istante una quantit\'a scalare. Un campo vettoriale \'e invece una funzione
  che associa ad ogni coppia di posizione ed istante una quantit\'a vettoriale.
  
  \mydefinition{Campo elettrico}
  Fissata l'origine del campo elettrico e considerata una ``carica di prova''
  piccola a piacere (cos\'i da non modificare l'origine del campo elettrico),
  si definisce
  \begin{equation}
    \label{eq:campo-elettrico}
    \begin{gathered}
      \vec{E}(\vec{r}) \equiv \frac{\vec{F}(\vec{r})}{q_0} =
      \frac{1}{4\pi\varepsilon_0}\frac{q}{r^2}\hat{r}
    \end{gathered}
  \end{equation}

  \mytheorem{Somma di campi elettrici}
  Il campo elettrico in un sistema di molte cariche elettriche poste ad una
  distanza, si calcola rispettivamente nel caso discreto
  (con $\vec{r}_i$ che indica le posizioni delle cariche)
  e nel caso continuo (con $\Omega \subset \reals^3$)attraverso:
  \begin{equation}
    \label{eq:somma-campi-elettrici}
    \begin{gathered}
      \vec{E}(\vec{r}) =
      \frac{1}{4\pi\varepsilon_0}
      \sum_{i=1}^N \left(
        q_i\cdot\frac{\vec{r} - \vec{r}_i}{|\vec{r} - \vec{r}_i|^3}
      \right) \\
      \vec{E}(\vec{r}) =
      \frac{1}{4\pi\varepsilon_0}
      \int_{\Omega}\left(
        \text{d}q\cdot\frac{\vec{r}}{|\vec{r}|^3}
      \right) \\
    \end{gathered}
  \end{equation}

  \mysection{Distribuzioni continue di carica}
  \begin{equation}
    \label{eq:densita-di-carica}
    \begin{gathered}
      (\text{Volume}) \quad \text{d}q = \boldsymbol{\rho}~\text{d}x\text{d}y\text{d}z = \boldsymbol{\rho}~\text{d}\tau \\
      (\text{Superficie}) \quad \text{d}q = \boldsymbol{\sigma}~\text{d}x\text{d}y = \boldsymbol{\sigma}~\text{d}\Sigma \\
      (\text{Lineare}) \quad \text{d}q = \boldsymbol{\lambda}~\text{d}l
    \end{gathered}
  \end{equation}

  \mytheorem{Filo carico}
  Il campo elettrico a distanza $z$ da un filo rettilineo infinito carico
  \begin{equation}
    \label{eq:filo-infinito}
    E(z) = \frac{1}{2\pi\varepsilon_0}\frac{\lambda}{z}
  \end{equation}

  \mytheorem{Anello carico}
  Il campo elettrico a distanza $z$ dal centro di un anello sottile carico di
  raggio $r$
  \begin{equation}
    \label{eq:anello-sottile-carico}
    E(z) = \frac{\lambda}{2\varepsilon_0}\left(
      \frac{rz}{(r^2+z^2)^{\frac{3}{2}}}
    \right)
  \end{equation}
  nel limite in cui $r\to 0$ l'anello degenera in un punto e da come ci si
  aspetta vale (\ref{eq:campo-elettrico})
  \begin{equation*}
    E(z) = \frac{Q}{4\pi\varepsilon_0}\frac{z}{(r^2+z^2)^{\frac{3}{2}}}, \quad
    Q = 2\pi r \lambda
  \end{equation*}


  \mytheorem{Disco carico}
  Il campo elettrico a distanza $z$ dal centro di un disco sottile carico di
  raggio $R$
  \begin{equation}
    \label{eq:disco-sottile-carico}
    E_R(z) = \frac{\sigma}{2\varepsilon_0}\left(
      1 - \frac{z}{\sqrt{z^2 + R^2}}
    \right)
  \end{equation}
  nel limite in cui il $R \to \infty$ il disco degenera in un piano infinito
  per il quale vale il limite dell'espressione precedente e dunque
  \begin{equation}
    \label{eq:piano-infinito-carico}
    E_{R\to\infty}(z) = \frac{\sigma}{2\varepsilon_0}
  \end{equation}

  \mysection{Flusso}
  \begin{equation}
    \label{eq:flusso}
    \varPhi_S(\vec{v}) = \int_S \vec{v}\cdot\hat{n}\text{d}S
  \end{equation}
  
  \mytheorem{Flusso uscente generato da una carica puntiforme su una superficie
    sferica}
  \begin{equation}
    \label{eq:flusso-elettrico-semplice}
    \varPhi_S(\vec{E}) = q/\varepsilon_0
  \end{equation}
  \begin{equation*}
    \myproof{
      \begin{gathered}
        \varPhi_S(\vec{E}) =
        \int_S \vec{E}\cdot\hat{n}\text{d}S =
        \int_S E\text{d}S =
        E \int_S \text{d}S =
        E \cdot (4\pi r^2) \\ =
        \frac{1}{4\pi\varepsilon_0}\frac{q}{r^2} \cdot (4\pi r^2) =
        q/\varepsilon_0
      \end{gathered}}
  \end{equation*}

  \mytheorem{Teorema di Gauss}
  Il flusso di un campo elettrico attraverso una superficie \underline{chiusa}
  dipende unicamente dalla carica complessiva $Q$ racchiusa
  \begin{equation}
    \label{th:gauss}
    \varPhi_S(\vec{E}) = Q / \varepsilon_0
  \end{equation}
  \begin{equation*}
    \myproof{
      \begin{gathered}
        \text{(Parte 1)} \quad \varPhi_S(\vec{E}) =
        \oint_S \vec{E}\cdot\hat{n}\text{d}S = \\ =
        \frac{q}{4\pi\varepsilon_0} \oint_S \frac{\hat{r}\cdot\hat{n}}{r^2}\cdot\text{d}S =
        \frac{q}{4\pi\varepsilon_0} \oint_S \text{d}\Omega,
        \quad \oint_S \text{d}\Omega = 4\pi\\ =
        \frac{q \cdot 4\pi}{4\pi\cdot\varepsilon_0} = \frac{q}{\varepsilon_0} \\
        \text{(Parte 2)} \quad \varPhi_S\left(\sum_i\vec{E}_i\right) =
        \oint_S \left(\sum_i\vec{E}_i\right)\text{d}\vec{S} \\ =
        \sum_i \left(\oint_S\vec{E}_i\text{d}\vec{S}\right) =
        \sum_i q_i/\varepsilon_0 \\
        \text{(Parte 3)} \quad q \not \in S \implies \#in = \#out\\
        \left[\text{d}\varPhi^{(out)} \propto \text{d}\Omega\right] ~ \wedge ~
        \left[\text{d}\varPhi^{(in)} \propto -\text{d}\Omega\right] \\
        \left[d\Omega^{in} = d\Omega^{out}\right] ~ \wedge ~
        \left[d\varPhi_{S} = d\varPhi_{S}^{(out)} + d\varPhi_{S}^{(in)}\right] \\
        d\varPhi_{S} = 0 \implies \varPhi_{S} = 0
      \end{gathered}
    }
  \end{equation*}

\end{multicols}

\end{document}