\documentclass[8pt, reqno, hidelinks]{amsart}
%\documentclass[8ptpt]{report}
\usepackage[margin=1.0cm]{geometry}
\usepackage[utf8]{inputenc}
\usepackage{multicol, wrapfig, blindtext, booktabs}
\usepackage{amsmath, amsthm, amsfonts, amssymb, mathrsfs, physics}
\usepackage{stackrel}
\usepackage{dsfont}
\usepackage{graphicx}
\usepackage{booktabs}
\usepackage{minted}
\usepackage{hyperref, bookmark}
\usepackage{cancel}
\usepackage{extarrows}
\usepackage{pagecolor}
\usepackage{tikz}
\usepackage{tikz-cd}
\usepackage{svg}
\usepackage{subcaption}
\usepackage{mathtools}
\usepackage{centernot}
\usepackage{xcolor}
\DeclarePairedDelimiter{\ceil}{\lceil}{\rceil}
\newenvironment{Figure}
  {\par\medskip\noindent\minipage{\linewidth}}
  {\endminipage\par\medskip}

% Standard Sets
\newcommand{\naturals}{\mathbb{N}}
\newcommand{\whole}{\mathbb{Z}}
\newcommand{\rationals}{\mathbb{Q}}
\newcommand{\reals}{\mathbb{R}}
\newcommand{\complex}{\mathbb{C}}
\newcommand{\field}{\mathbb{K}}
\newcommand{\cm}{\text{CM}}
\newcommand{\due}[1]{{^\star}#1}

\definecolor{blue}{HTML}{0044cc}
\definecolor{red}{HTML}{ff3333}
\definecolor{light-green}{HTML}{cdff99}
\definecolor{light-purple}{HTML}{ffcccc}
\definecolor{light-orange}{HTML}{ffdd99}



%\newcommand{\mychapter}[1]{\section*{\texorpdfstring{\colorbox{light-purple}{#1}}{#1}}}
\newcommand{\mychapter}[1]{\section*{\texorpdfstring{\fbox{#1}}{#1}}}
\newcommand{\mysection}[1]{\section*{\texorpdfstring{\colorbox{light-orange}{#1}}{#1}}}
\newcommand{\mydefinition}[1]{\subsection*{\textcolor{red}{#1}}}
\newcommand{\mytheorem}[1]{\subsection*{\textcolor{blue}{#1}}}
\newcommand{\myparagraph}[1]{\paragraph{\textbf{#1}}}
\newcommand{\myproof}[1]{\tiny{\textcolor{gray}{#1}}}



\begin{document}
\mychapter{Appunti}

\begin{multicols}{3}
  \mydefinition{Forze di natura elettrica}
  La forza elettrica \'e una formza con azione a distanza osservata inizialmente
  da esperienze comuni come l'attrazione di pezzetti di carta ad una bachetta di
  vetro strofinata con della pelle di gatto (vivo sperabilmente).

  \mydefinition{Ipotesi di Franklin (da revisionare)}
  Esistono 2 classi di materiali: positivi (come il vetro) e negativi
  (come la bachelite).

  \mydefinition{Modello atomico di Thomson}
  Il modello atomico di Thomson (1898) noto come modello a panettone prevede un
  corpo carico positivamente popolato da corpi minori di carica negativa
  all'interno. Il modello venne superato a breve da Rutherford (1909)
  attraverso le sue osservazioni dall'esperimento di scattering; esso spiega
  la deflessione di particelle cariche positivamente con un modello planetario
  dell'atomo che coinvolge un denso nucleo positivo a cui orbitano attorno le
  cariche negative.

  \mydefinition{Ordini di grandezza}
  La grandezza di un atomo \'e nell'ordine di $10^{-10}m$ mentre il suo nucleo
  nell'ordine di $10^{-15}m$. A queste scale microscopiche si ricorre usualmente
  ad una scala pi\'u conveniente detta Angstrom $1\mathring{A} = 10^{-10}m$

  \mydefinition{La sostanza della materia}
  La massa del protone $m_p\approx 1.6725\cdot 10^{-27} kg$,
  la massa dell'elettrone $m_e \approx 9.1094 \cdot 10^{-31} kg$ e
  la massa del neutrone $m_n\approx 1.6748\cdot 10^{-27} kg$.
  La carica dell'elettrone  $e = 1.69 \cdot 10^{-19} C$ viene detta anche
  \textit{carica elementare} siccome ogni altro corpo carico deve essere un
  multiplo intero di questa.
  

  \mysection{Metodi di elettrizzazione}
  \mydefinition{Triboelettrico}
  Fenomeno elettrico che consiste nel trasferimento di cariche elettriche tra
  materiali neutri diversi (di cui almeno uno isolante) quando vengono
  \textbf{strofinati} tra di loro.

  \mydefinition{Induzione}
  Il fenomeno per cui la carica elettrica all'interno di un oggetto viene
  ri-distribuita a causa della presenza di un altro oggetto carico nelle
  \textbf{vicinanze}.

  \mydefinition{Conduzione}
  Fenomeno elettrico che consiste nel trasferimento di cariche elettriche tra
  un materiale carico ed uno neutro quando vengono messi a \textbf{contatto}
  fra di loro.

  \mydefinition{Legge di conservazione della carica}
  In un sistema isolato la carica complessiva si conserva.
  \begin{equation}
    \label{eq:conservazione-carica}
    \sum_i \frac{\Delta q_i(t)}{\Delta t} = 0
  \end{equation}

  \mydefinition{Legge di coulomb}
  \begin{equation}
    \label{eq:legge-coulomb}
    \vec{f}_{12} = k\frac{q_1q_2}{r^2}\hat{r}_{12}
  \end{equation}
  \mydefinition{Campo scalare e vettoriale}
  \mydefinition{Campo elettrico}
\end{multicols}

\end{document}