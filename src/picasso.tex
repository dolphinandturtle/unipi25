\documentclass[8pt, reqno, hidelinks]{amsart}
%\documentclass[8ptpt]{report}
\usepackage[margin=1.0cm]{geometry}
\usepackage[utf8]{inputenc}
\usepackage{multicol, wrapfig, blindtext, booktabs}
\usepackage{amsmath, amsthm, amsfonts, amssymb, mathrsfs, physics}
\usepackage{stackrel}
\usepackage{dsfont}
\usepackage{graphicx}
\usepackage{booktabs}
\usepackage{minted}
\usepackage{hyperref, bookmark}
\usepackage{cancel}
\usepackage{extarrows}
\usepackage{pagecolor}
\usepackage{tikz}
\usepackage{tikz-cd}
\usepackage{svg}
\usepackage{subcaption}
\usepackage{mathtools}
\usepackage{centernot}
\usepackage{xcolor}
\DeclarePairedDelimiter{\ceil}{\lceil}{\rceil}
\newenvironment{Figure}
  {\par\medskip\noindent\minipage{\linewidth}}
  {\endminipage\par\medskip}

% Standard Sets
\newcommand{\naturals}{\mathbb{N}}
\newcommand{\whole}{\mathbb{Z}}
\newcommand{\rationals}{\mathbb{Q}}
\newcommand{\reals}{\mathbb{R}}
\newcommand{\complex}{\mathbb{C}}
\newcommand{\field}{\mathbb{K}}
\newcommand{\cm}{\text{CM}}
\newcommand{\due}[1]{{^\star}#1}

\definecolor{blue}{HTML}{0044cc}
\definecolor{red}{HTML}{ff3333}
\definecolor{light-green}{HTML}{cdff99}
\definecolor{light-purple}{HTML}{ffcccc}
\definecolor{light-orange}{HTML}{ffdd99}



%\newcommand{\mychapter}[1]{\section*{\texorpdfstring{\colorbox{light-purple}{#1}}{#1}}}
\newcommand{\mychapter}[1]{\section*{\texorpdfstring{\fbox{#1}}{#1}}}
\newcommand{\mysection}[1]{\section*{\texorpdfstring{\colorbox{light-orange}{#1}}{#1}}}
\newcommand{\mydefinition}[1]{\subsection*{\textcolor{red}{#1}}}
\newcommand{\mytheorem}[1]{\subsection*{\textcolor{blue}{#1}}}
\newcommand{\myparagraph}[1]{\paragraph{\textbf{#1}}}
\newcommand{\myproof}[1]{\tiny{\textcolor{gray}{#1}}}

\usepackage[italian]{babel}

\title{"Lezioni di Fisica Generale 2 by Luigi E. Picasso"}

\begin{document}
\maketitle

\mychapter{Keywords}
\begin{multicols}{3}
  Legge di Coulomb. Convenzione di Franklin. Sistema di Gauss. Principio di sovvrapposizione.
  Campo elettrico. Sorgente del campo. Teorie di campo. Linee di forza. Densit\'a di carica.
  Integrale di Riemann. Coordinate sferiche. Angolo solido. Distribuzione di volume.
  Distribuzione superficiale. Distribuzione lineare. Superifice orientabile. Flusso.
  Portata del fluido.
\end{multicols}

\mychapter{Pages}
\begin{multicols}{3}
  \begin{enumerate}
  \item Introduzione storica. Legge di Coulomb e interazioni a grandi distanze.
  \item Convenzione di Franklin. Determinazione del fattore di proporzionalit\'a della legge di Coulomb.
    Sistema di Gauss.
  \item Scelta di sistema di u.m. Principio di sovvrapposizione e ipotesi per la legge di Coulomb.
  \item Paragone con la legge di gravitazione. Conseguenze non fisiche della legge di Coulomb. Il campo elettrico e la proporzionalit\'a alla forza.
  \item Definizione operativa del campo elettrico e preludio alle teorie di campo.
  \item Linee di forza del campo e i loro punti singolari. Ipotesi di continuo e densit\'a di carica.
  \item Scala fisica del volume ``piccolo''. Carica e campo elettrico totali.
  \item Sistemi di coordinate per calcolo integrale sul volume.
  \item Stabilit\'a del campo elettrico generato da un volume. Distribuzioni degeneri e le loro singolarit\'a.
  \item Propriet\'a delle superfici e flusso.
  \end{enumerate}
\end{multicols}

\mychapter{Propositions}
\begin{multicols}{3}
  \mydefinition{Legge di Coulomb}
  \begin{equation}
    \vec{F} \equiv k \frac{q_1q_2}{|\vec{r}|^2}\hat{r}
  \end{equation}

  \mydefinition{Campo elettrico}
  Con $q_1$ \textit{sorgente del campo}.
  \begin{equation}
    \vec{E}_1(\vec{r}) \equiv k \frac{q_1}{|\vec{r} - \vec{r}_1|^3}(\vec{r} - \vec{r}_1)
  \end{equation}
  
  \mydefinition{Flusso}
  \begin{equation}
    \Phi_S(\vec{F}) \equiv \int_S \vec{F} \cdot \text{d}\vec{S} \quad \vec{S} \equiv S\hat{n}
  \end{equation}
  
\end{multicols}

\end{document}
