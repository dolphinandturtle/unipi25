\documentclass[8pt, reqno, hidelinks]{amsart}
%\documentclass[8ptpt]{report}
\usepackage[margin=1.0cm]{geometry}
\usepackage[utf8]{inputenc}
\usepackage{multicol, wrapfig, blindtext, booktabs}
\usepackage{amsmath, amsthm, amsfonts, amssymb, mathrsfs, physics}
\usepackage{stackrel}
\usepackage{dsfont}
\usepackage{graphicx}
\usepackage{booktabs}
\usepackage{minted}
\usepackage{hyperref, bookmark}
\usepackage{cancel}
\usepackage{extarrows}
\usepackage{pagecolor}
\usepackage{tikz}
\usepackage{tikz-cd}
\usepackage{svg}
\usepackage{subcaption}
\usepackage{mathtools}
\usepackage{centernot}
\usepackage{xcolor}
\DeclarePairedDelimiter{\ceil}{\lceil}{\rceil}
\newenvironment{Figure}
  {\par\medskip\noindent\minipage{\linewidth}}
  {\endminipage\par\medskip}

% Standard Sets
\newcommand{\naturals}{\mathbb{N}}
\newcommand{\whole}{\mathbb{Z}}
\newcommand{\rationals}{\mathbb{Q}}
\newcommand{\reals}{\mathbb{R}}
\newcommand{\complex}{\mathbb{C}}
\newcommand{\field}{\mathbb{K}}
\newcommand{\cm}{\text{CM}}
\newcommand{\due}[1]{{^\star}#1}

\definecolor{blue}{HTML}{222255} %#0044cc
\definecolor{red}{HTML}{552222} %#ff3333
\definecolor{light-green}{HTML}{cdff99}
\definecolor{light-purple}{HTML}{ffcccc}
\definecolor{light-orange}{HTML}{ffeec0} %#ffdd99



%\newcommand{\mychapter}[1]{\section*{\texorpdfstring{\colorbox{light-purple}{#1}}{#1}}}
\newcommand{\mychapter}[1]{\section*{\texorpdfstring{\fbox{#1}}{#1}}}
\newcommand{\mysection}[1]{\section*{\texorpdfstring{\colorbox{light-orange}{#1}}{#1}}}
\newcommand{\mydefinition}[1]{\subsection*{\textcolor{red}{#1}}}
\newcommand{\mytheorem}[1]{\subsection*{\textcolor{blue}{#1}}}
\newcommand{\myparagraph}[1]{\paragraph{\textbf{#1}}}
\newcommand{\myproof}[1]{\tiny{\textcolor{gray}{#1}}}

\usepackage[italian]{babel}

\title{"Lezioni di Fisica Generale 2 by Luigi E. Picasso"}

\begin{document}
\maketitle

\mychapter{Keywords}
\begin{multicols}{3}
  Legge di Coulomb. Convenzione di Franklin. Sistema di Gauss. Principio di sovvrapposizione.
  Campo elettrico. Sorgente del campo. Teorie di campo. Linee di forza. Densit\'a di carica.
  Integrale di Riemann. Coordinate sferiche. Angolo solido. Distribuzione di volume.
  Distribuzione superficiale. Distribuzione lineare. Superifice orientabile. Nastro di Moebius. Flusso.
  Portata del fluido. Teorema di Gauss. Divergenza. Densit\'a di flusso. Teorema della divergenza.
  Prima equazione di Maxwell. Legge di densit\'a. Carattere integrale. Carattere locale. Simmetria sferica.
  Tubo di flusso. Isometrie. Punto di vista attivo. Campo scalare. Campo vettoriale. Campi di forze.
  Campi di velocit\'a. Vettori assiali. Spazio omogeneo ed isotropo. Etere. Sotto-gruppo delle isometrie.
  Gruppo di invarianza delle sorgenti. Azione del gruppo. Stabilizzatore. Auto-vettore. Simmetria cilindrica.
  Simmetria piana. Coniugazione di carica. Trasformazione di invarianza. Dipolo. Superficie di Gauss. Condensatore.
  Invarianza di scala. Campo centrale. Conservativo. Elemento di linea orientato. Gradiente. Teorema di Schwartz. Rotore.
  Irrotazionale. Momento di dipolo. Quadripolo. Termini di multipolo. Campo di dipolo. Dipolo puntiforme.
\end{multicols}

\mychapter{Pages}
\begin{multicols}{3}
  \myparagraph{1}
  La scoperta dell'elettrone e delle classi di equivalenza delle cariche. La legge di Coulomb e le scale esplorate.
  \myparagraph{2}
  Additivit\'a delle cariche e sistemi di misura.
  \myparagraph{3}
  Scelta del sistema di misura nell'ambito teorico e pratico. Principio di sovvrapposizione per corpi carichi macroscopici.
  \myparagraph{4}
  Carattere matematico della legge di Coulomb. Il campo elettrico e proporzionalit\'a della forza alla carica.
  \myparagraph{5}
  Definizione operativa del campo elettrico e transizione dalla teoria meccanica alle teorie di campo.
  \myparagraph{6}
  Linee di forza del campo e i loro punti singolari. Ipotesi di continuo e densit\'a di carica.
  \myparagraph{7}
  Scala fisica del volume ``piccolo'' e carica totale.
  \myparagraph{8}
  Sistemi di coordinate per calcolo integrale sul volume.
  \myparagraph{9}
  Stabilit\'a del campo elettrico generato da un volume. Distribuzioni degeneri e le loro singolarit\'a.
  \myparagraph{10}
  Propriet\'a delle superfici. Definizione di flusso ed analogia con la meccanica dei fluidi.
  \myparagraph{11}
  Flusso del campo elettrico uscente dalla sfera. Spiegazione intuitiva del teorema di Gauss. Definizione di divergenza.
  \myparagraph{12}
  Additivit\'a del flusso rispetto al volume e relazione con la divergenza. Teorema della divergenza.
  \myparagraph{13}
  Divergenza del campo elettrico di una carica puntiforme. Teorema di Gauss.
  \myparagraph{14}
  Formulazione locale del teorema di Gauss. La legge di densit\'a.
  \myparagraph{15}
  Simmetrie e teorema di Gauss. Isometria su una densit\'a di carica.
  \myparagraph{16}
  Teorema di trasporto del campo elettrico, dimostrazione, casi particolari ed densit\'a di carica invariante per isometrie.
  \myparagraph{17}
  Isotropia della legge di Coulomb. Gruppo di invarianza delle sorgenti, azione del gruppo sui punti.
  Esempio su distribuzione a simmetria sferica.
  \myparagraph{18}
  Esempio su distribuzioni a simmetria cilindrica e a simmetria piana. Coniugazione di carica.
  \myparagraph{19}
  Campo elettrico delle distribuzioni a simmetria sferica e simmetria cilindrica.
  \myparagraph{20}
  Campo elettrico della distribuzione a simmetria piana.
  \myparagraph{21}
  Sistema del condensatore.
  \myparagraph{22}
  Invarianza di scala. Potenziale del campo elettrico. Propriet\'a fondamentali dei campi conservativi: circuitazione nulla.
  \myparagraph{23}
  Propriet\'a fondamentali dei campi conservativi: potenziale definito a meno di una costante additiva, rotore nullo
  (con analogia alla meccanica dei fluidi) e relazione con i campi conservativi, e superfici equipotenziali.
  \myparagraph{24}
  Potenziale nel sistema di Gauss. Misure di potenziale elettrico e unit\'a del campo elettrico.
  Esempio: modello di Rutherford dell'atomo di idrogeno, potenziale dovuto ad una sfera carica.
  \myparagraph{25}
  Esempi: potenziale dovuto ad una sfera con distribuzione di carica superficiale e sferica, potenziale del condensatore piano,
  filo rettilineo indefinito con distribuzione lineare costante.
  \myparagraph{26}
  Potenziale del campo elettrico a grandi distanze, sviluppo del potenziale in serie di potenze e momento di dipolo.
  \myparagraph{27}
  Campo di dipolo e indipendenza dalla scelta del polo per sistemi di carica nulla, momento intrinseco di dipolo e limite del dipolo puntiforme.
\end{multicols}

\mychapter{Propositions}
\begin{multicols}{3}
  \mydefinition{Legge di Coulomb}
  \begin{equation}
    \vec{F} \equiv k \frac{q_1q_2}{|\vec{r}|^2}\hat{r}
  \end{equation}

  \mydefinition{Campo elettrico}
  Con $q_1$ \textit{sorgente del campo}.
  \begin{equation}
    \vec{E}_1(\vec{r}) \equiv k \frac{q_1}{|\vec{r} - \vec{r}_1|^3}(\vec{r} - \vec{r}_1)
  \end{equation}
  
  \mydefinition{Flusso}
  \begin{equation}
    \Phi_S(\vec{F}) \equiv \int_S \vec{F} \cdot \text{d}\vec{S} \quad \vec{S} \equiv S\hat{n} \\
  \end{equation}
  il flusso di una velocit\'a pu\'o essere interpretato come il volume che attraversa la superficie per unit\'a di tempo.

  \mydefinition{Divergenza}
  \begin{equation}
    \text{div}~\vec{F} \equiv \lim_{\Delta V \to 0} \frac{\Phi_V(\vec{F})}{\Delta V}
  \end{equation}
  
  \mytheorem{Teorema della divergenza}
  \begin{equation}
    \text{div}~\vec{F} = \pdv{F_x}{x} + \pdv{F_y}{y} + \pdv{F_z}{z}
  \end{equation}

  \mytheorem{Teorema di Gauss}
  Il flusso del campo elettrico uscente da una superficie chiusa \'e proporzionale alla carica totale contenuta dentro la superficie.
  \begin{equation}
    \Phi_S(\vec{E}) = 4\pi k Q^{\text{int}}
  \end{equation}

\end{multicols}

\end{document}
