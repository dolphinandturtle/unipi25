\documentclass[8pt, reqno, hidelinks]{amsart}
%\documentclass[8ptpt]{report}
\usepackage[margin=1.0cm]{geometry}
\usepackage[utf8]{inputenc}
\usepackage{multicol, wrapfig, blindtext, booktabs}
\usepackage{amsmath, amsthm, amsfonts, amssymb, mathrsfs, physics}
\usepackage{stackrel}
\usepackage{dsfont}
\usepackage{graphicx}
\usepackage{booktabs}
\usepackage{minted}
\usepackage{hyperref, bookmark}
\usepackage{cancel}
\usepackage{extarrows}
\usepackage{pagecolor}
\usepackage{tikz}
\usepackage{tikz-cd}
\usepackage{svg}
\usepackage{subcaption}
\usepackage{mathtools}
\usepackage{centernot}
\usepackage{xcolor}
\DeclarePairedDelimiter{\ceil}{\lceil}{\rceil}
\newenvironment{Figure}
  {\par\medskip\noindent\minipage{\linewidth}}
  {\endminipage\par\medskip}

% Standard Sets
\newcommand{\naturals}{\mathbb{N}}
\newcommand{\whole}{\mathbb{Z}}
\newcommand{\rationals}{\mathbb{Q}}
\newcommand{\reals}{\mathbb{R}}
\newcommand{\complex}{\mathbb{C}}
\newcommand{\field}{\mathbb{K}}
\newcommand{\cm}{\text{CM}}
\newcommand{\due}[1]{{^\star}#1}

\definecolor{blue}{HTML}{222255} %#0044cc
\definecolor{red}{HTML}{552222} %#ff3333
\definecolor{light-green}{HTML}{cdff99}
\definecolor{light-purple}{HTML}{ffcccc}
\definecolor{light-orange}{HTML}{ffeec0} %#ffdd99



%\newcommand{\mychapter}[1]{\section*{\texorpdfstring{\colorbox{light-purple}{#1}}{#1}}}
\newcommand{\mychapter}[1]{\section*{\texorpdfstring{\fbox{#1}}{#1}}}
\newcommand{\mysection}[1]{\section*{\texorpdfstring{\colorbox{light-orange}{#1}}{#1}}}
\newcommand{\mydefinition}[1]{\subsection*{\textcolor{red}{#1}}}
\newcommand{\mytheorem}[1]{\subsection*{\textcolor{blue}{#1}}}
\newcommand{\myparagraph}[1]{\paragraph{\textbf{#1}}}
\newcommand{\myproof}[1]{\tiny{\textcolor{gray}{#1}}}



\begin{document}

\mychapter{Corso}
\begin{multicols}{3}

  \mysection{Orario}
  \mydefinition{Lezione}
  \begin{enumerate}
  \item Luned\'i dalle 8:30 alle 10:30
  \item Marted\'i dalle 10:30 alle 12:30
  \item Gioved\'i dalle 10:30 alle 12:30
  \end{enumerate}
  \mydefinition{Ricevimento} (edificio C, primo piano, 161)
  \begin{enumerate}
  \item Luned\'i pomeriggio (16:30)
  \item Venerd\'i pomeriggio
  \end{enumerate}

  \mysection{Programma}
  \mydefinition{Primo semestre}
  Si studia la meccanica classica, una teoria fisica nata dal trattato di
  meccanica analitica (razionale) pubblicato da Lagrange. Da uno studio
  sistematico della meccanica Newtoniana e dei suoi principi si espone la
  teoria meccanica di Lagrange e successivamente la teoria di Hamilton le cui
  formulazioni saranno spunto per un campo della matematica detto ``calcolo
  delle variazioni'' e del cosidetto ``principio variazionale''.

  \begin{equation}
    \begin{gathered}
      \text{Newtoniana} \to
      \text{Lagrangiana} \to
      \text{Hamiltoniana} \\
      \to \text{Quantistica}
    \end{gathered}
  \end{equation}

  Successivamente si studier\'a il legame fra le simmetrie in fisica e le
  quantit\'a conservate.

  \begin{equation}
    \text{Simmetrie}
    \begin{matrix}
      \xrightarrow{\text{Lagrange}} \\
      \xleftarrow{\text{Hamilton}}
    \end{matrix}
    \text{Conservazioni}
  \end{equation}

  Si affronteranno studi sul corpo rigido introducendo il formalismo
  tensoriale. Si studieranno le piccole oscillazioni ovvero la linearizzazione
  dei moti attorno ai punti di equilibrio stabili.

  \mydefinition{Secondo semestre}
  Si studia la relativit\'a ristretta in maniera pi\'u approfondita attraverso
  gli strumenti sviluppati nel primo semestre. Infine attraverso la meccanica
  di Hamilton si studier\'a la meccanica microscopica e come essa si possa
  applicare alla termodinamica.

  \mysection{Esame}
  L'esame scritto \'e formato da 4 problemi: A1, A2, R, S (dove A1 \'e
  solitamente un problema sulle piccole oscillazioni). Ogni problema \'e
  composto da 4 domande di difficolt\'a crescente per un totale di 16 punti
  l'uno. Successivamente se uno \'e soddisfatto dell'esame e a meno che i
  docenti non lo richiedano, il voto pu\'o essere finalizzato senza il
  superamento dell'esame orale.

  \mysection{Bibliografia}
  \begin{itemize}
  \item ``Landau 1''
  \item ``Goldsten''
  \item ``Arnold''
  \end{itemize}

\end{multicols}

\mychapter{Appunti}
\begin{multicols}{3}
  \mydefinition{Punto materiale}
  Il sistema fisico pi\'u semplice possibile descritto univocamente dalla sua
  posizione $\vec{x}(t)$, la cui evoluzione temporale si esprime con le
  quantit\'a cinematiche, rispettivamente: \textit{velocit\'a},
  \textit{accelerazione}, \textit{quantit\'a di moto},
  \textit{momento angolare} ed \textit{energia cinetica}:
  \begin{equation}
    \label{eq:quantita-cinematiche}
    \begin{gathered}
      \vec{v} \equiv \dv{\vec{x}}{t} = \dot{\vec{x}}
      \quad
      \vec{a} \equiv \dv{\vec{v}}{t} = \ddot{\vec{x}} \\
      \vec{p} \equiv m \vec{v}
      \quad
      \vec{l} \equiv \vec{x} \wedge \vec{p}
      \quad
      T \equiv \frac{1}{2}m|\vec{v}|^2
    \end{gathered}
  \end{equation}

  \mydefinition{Principi della dinamica}
  \begin{enumerate}
  \item Se siamo in un sistema di riferimento inerziale e non agiscono forze
    esterne sul punto materiale allora esso compie un moto rettilineo uniforme
    (o in quiete).
  \item Se agisce una forza $F$ sul punto materiale allora
    $F = m\vec{a} = \dot{\vec{p}}$
  \item In un sistema isolato la forza che un punto materiale esercita su un
    altro \'e uguale in modulo e direzione ed opposto in verso rispetto alla
    forza che l'altro esercita su di esso.
  \end{enumerate}

  \mytheorem{Teorema dell'energia cinetica}
  La variazione di energia cinetica \'e pari al lavoro elementare compiuto
  dalla forza.
  \begin{equation}
    \label{eq:teorema-cinetica}
    \text{d}T = \vec{F} \cdot \text{d}\vec{x}
  \end{equation}
  \begin{equation*}
    \myproof{
    \begin{gathered}
      \dv{T}{t} =
      \dv{t}\left(\frac{1}{2}m(\vec{v}\cdot\vec{v})\right) =
      \frac{1}{2}m\dv{\vec{v}}{t}\cdot\vec{v} +
      \frac{1}{2}m\vec{v}\cdot\dv{\vec{v}}{t} \\ =
      m\dv{\vec{v}}{t}\cdot\vec{v} = \dv{\vec{p}}{t}\cdot\vec{v} =
      \vec{F} \cdot \vec{v} = \vec{F} \cdot \dv{\vec{x}}{t} \\
      \implies \text{d}T = \vec{F} \cdot \text{d}\vec{x}
    \end{gathered}}
  \end{equation*}

  % Da aggiornare
  \mytheorem{Teorema momento angolare}
  La derivata del momento angolare di un punto materiale rispetto ad un
  \textit{polo fisso} \'e il momento della forza
  \begin{equation}
    \label{eq:teorema-momento-angolare}
    \dv{\vec{l}}{t} = \vec{x} \wedge \vec{F}
  \end{equation}
  \begin{equation*}
    \myproof{
      \begin{gathered}
        \dv{l}{t} = \dv{t}\left(\vec{x} \wedge \vec{p}\right) =
        \dot{\vec{x}} \wedge \vec{p} + \vec{x} \wedge \vec{F} =
        \vec{v} \wedge m\vec{v} + \vec{x} \wedge \vec{F} \\
        = m\cancelto{0}{(\vec{v} \wedge \vec{v})} + \vec{x} \wedge \vec{F} =
        \vec{x} \wedge \vec{F}
      \end{gathered}}
  \end{equation*}

  \mytheorem{Campi conservativi}
  \begin{equation}
    \label{eq:campi-conservativi}
    \begin{gathered}
      \pdv{f}{x} \Big)_y = \lim_{\varepsilon \to 0} \frac{f(x + \varepsilon, y) - f(x, y)}{\varepsilon}\\
      \vec{F} = -\vec{\nabla}U(\vec{x}, t) = \left(
        \pdv{U}{x}, \pdv{U}{y}, \pdv{U}{z}
      \right) \\
      F_x = -\pdv{U}{x} \quad F_y = -\pdv{U}{y} \quad F_z = -\pdv{U}{z} \\
      f(x + \text{d}x, y + \text{d}y) \approx f(x, y) + \pdv{f}{x}\text{d}x
      + \pdv{f}{y}\text{d}y \\
      = f(x, y) + \text{d}f(x, y) = \vec{\nabla}f \cdot \text{d}\vec{x}
    \end{gathered}
  \end{equation}

  \mytheorem{Conservazione dell'energia meccanica}
  \begin{equation}
    \label{eq:conservazione-meccanica}
    \begin{gathered}
      U(x, y, z) \implies \dv{t}\left(T + U\right) = 0 \\
      U(x, y, z, t) \implies \dv{t}\left(T + U\right) = \pdv{U}{t}
    \end{gathered}
  \end{equation}

  \mytheorem{Dinamica dei sistemi}
  Centro di massa \\
  Teoremi di K\"onig


\end{multicols}



\end{document}
