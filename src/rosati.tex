\documentclass[8pt, reqno, hidelinks]{amsart}
%\documentclass[8ptpt]{report}
\usepackage[margin=1.0cm]{geometry}
\usepackage[utf8]{inputenc}
\usepackage{multicol, wrapfig, blindtext, booktabs}
\usepackage{amsmath, amsthm, amsfonts, amssymb, mathrsfs, physics}
\usepackage{stackrel}
\usepackage{dsfont}
\usepackage{graphicx}
\usepackage{booktabs}
\usepackage{minted}
\usepackage{hyperref, bookmark}
\usepackage{cancel}
\usepackage{extarrows}
\usepackage{pagecolor}
\usepackage{tikz}
\usepackage{tikz-cd}
\usepackage{svg}
\usepackage{subcaption}
\usepackage{mathtools}
\usepackage{centernot}
\usepackage{xcolor}
\DeclarePairedDelimiter{\ceil}{\lceil}{\rceil}
\newenvironment{Figure}
  {\par\medskip\noindent\minipage{\linewidth}}
  {\endminipage\par\medskip}

% Standard Sets
\newcommand{\naturals}{\mathbb{N}}
\newcommand{\whole}{\mathbb{Z}}
\newcommand{\rationals}{\mathbb{Q}}
\newcommand{\reals}{\mathbb{R}}
\newcommand{\complex}{\mathbb{C}}
\newcommand{\field}{\mathbb{K}}
\newcommand{\cm}{\text{CM}}
\newcommand{\due}[1]{{^\star}#1}

\definecolor{blue}{HTML}{222255} %#0044cc
\definecolor{red}{HTML}{552222} %#ff3333
\definecolor{light-green}{HTML}{cdff99}
\definecolor{light-purple}{HTML}{ffcccc}
\definecolor{light-orange}{HTML}{ffeec0} %#ffdd99



%\newcommand{\mychapter}[1]{\section*{\texorpdfstring{\colorbox{light-purple}{#1}}{#1}}}
\newcommand{\mychapter}[1]{\section*{\texorpdfstring{\fbox{#1}}{#1}}}
\newcommand{\mysection}[1]{\section*{\texorpdfstring{\colorbox{light-orange}{#1}}{#1}}}
\newcommand{\mydefinition}[1]{\subsection*{\textcolor{red}{#1}}}
\newcommand{\mytheorem}[1]{\subsection*{\textcolor{blue}{#1}}}
\newcommand{\myparagraph}[1]{\paragraph{\textbf{#1}}}
\newcommand{\myproof}[1]{\tiny{\textcolor{gray}{#1}}}

\usepackage[italian]{babel}


\begin{document}

\mychapter{Keywords}
\begin{multicols}{3}
  lunghezza,
  massa,
  temperatura,
  carica elettrica,
  intensit\'a di corrente elettrica,
  elettrostatica,
  elettrizzati o elettricamente carichi,
  forze elettriche,
  cariche elettriche positive/negative,
  elettroscopio a foglie,
  coduttori,
  isolanti o dielettrici,
  ionizzate,
  cariche puntiformi,
  legge di Coulomb,
  permittivit\'a o costante dielettrica,
  coulomb (unit\'a di misura),
  campo elettrico,
  carica di prova,
  intensit\'a del campo elettrico,
  linee di forza,
  rappresentazione di Faraday,
  principio di sovrapposizione,
  densit\'a di carica elettrica (lineare, superficiale e volumetrica),
  potenziale elettrostatico,
  superfici equipotenziali,
  tensione elettrica,
  condensatore piano,
  facce (o armature) del condensatore,
  dipolo elettrico,
  momento elettrico di dipolo,
  elettrone,
  elettronvolt (unit\'a di misura),
  energia di legame,
  stato solido/liquido/gassoso,
  fusione,
  solidificazione,
  evaporazione,
  liquefazione,
  sublimazione,
  legge della conservazione della massa,
  legge delle proporzioni definite,
  legge delle proporzioni multiple,
  legge di Avogadro,
  atomi,
  elemento chimico (o sostanza chimica semplice),
  mono/bi/tri/.../poli-atomico,
  peso atomico (o peso molecolare),
  raggio classico dell'elettrone,
  modello atomico di Rutherford,
  nucleo,
  numero atomico,
  angolo di diffusione,
  interazione nucleare,
  protone,
  isotopi,
  neutrone,
  gusci dell'atomo,
  ionizzato,
  ione negativo,
  energia di ionizzazione (dell'atomo),
  energia di legame (dell'elettrone),
  elettroni di valenza,
  struttura cristallina,
  mono/poli-cristalli,
  legame ionico (o eteropolare),
  legame covalente (o omopolare),
  conduttori metallici (o di prima classe),
  elettroni di conduzione (o elettroni liberi),
  corrente elettrica,
  conduttori elettrolitici (o di seconda classe),
  cationi,
  anioni,
  conduzione elettrolitica,
  conduttori gassosi,
  isolanti,
  semi-conduttori,
  generatori di forza elettromotrice,
  f.e.m (o forza elettromotrice),
  intensit\'a di corrente,
  densit\'a di corrente,
  amp\'ere (unit\'a di misura),
  verso convenzionale,
  galvanometro,
  amperometro,
  corrente continua,
  corrente alternata,
  principio di conservazione della carica elettrica,
  equazione di continuit\'a,
  legge di Ohm,
  resistenza,
  ohm (unit\'a di misura),
  resistenza specifica (o resistivit\'a),
  (super-)conduttivit\'a,
  coefficiente termico della resistenza,
  termometri (o pirometri) a resistenza,
  resistori campioni,
  reostati,
  approssimazione di elettroni indipendenti,
  approssimazione di elettroni liberi,
  effetto Joule,
  equivalente meccanico della chilocaloria,
  resistori (o resistenze) in serie/parallelo,
  conduttanze,
  resistenza interna,
  caduta interna di tensione,
  conduttanza interna,
  rete elettrica,
  nodo (di un circuito),
  rami (di un circuito),
  prima legge di Kirchhoff,
  maglia (di un circuito),
  seconda legge di Kirchhoff,
  ponte in equilibrio (o bilanciato),
  generatore adattato per il massimo trasferimento di potenza,
  metodo dei rami,
  reti elettriche lineari,
  metodo delle maglie,
  metodo dei nodi,
  equazioni delle maglie,
  equazioni dei nodi,
  auto-conduttanza,
  mutua conduttanza,
  corrente di eccitazione del nodo,
  teorema di sovvraposizione (per un circuito),
  generatore attivo/passivo,
  teorema di compensazione (per un circuito),
  teorema di Th\'evenin,
  teorema di Norton,
  teorema di reciprocit\'a (per un circuito),
  lavoro di estrazione,
  potenziale interno,
  potenziale di estrazione,
  effetto termoionico,
  effetto fotoelettrico,
  formula di Richardson,
  costante di Planck,
  effetto di Schottky,
  emissione a freddo,
  effetto Volta,
  prima legge di volta,
  catena regolarmente aperta,
  seconda legge di volta,
  serie voltaica,
  giunzione,
  effetto termoelettrico (o effetto di Seebeck),
  termocoppia (o coppia termoelettrica o pila termoelettrica),
  f.e.m termoelettrica,
  legge del metallo intermedio,
  effetto Peltier,
  coefficiente di Peltier,
  f.e.m di Peltier,
  effetto Thomson,
  effetto Thomson,
  coefficiente di Thomson,
  elettrodi,
  anodo,
  catodo,
  cella elettrolitica (o voltametro),
  elettroliti,
  soluzione elettrolitica,
  dissociazione elettrolitica,
  cationi,
  anioni,
  elettro-positivi/negativi,
  grado di dissociazione,
  pressione osmotica,
  elettroliti forti/deboli,
  mobilit\'a degli ioni,
  numeri di trasporto,
  conduttivit\'a equivalente,
  conduttivit\'a equivalente limite,
  elettrolisi,
  prima legge di Faraday,
  equivalente elettrochimico,
  seconda legge di Faraday,
  faraday,
  coulomb internazionale,
  pila di Volta,
  polarizzazione,
  pila primaria,
  pile secondarie,
  accumulatori,
  accumulatori al piombo,
  accumulatori al nichel-ferro,
  accumulatori al nichel-cadmio,
  accumulatori,
  raffinazione elettrolitica,
  galvanoplastica,
  galvanostegia,
  condensatori elettrolitici,
  teorema di Gauss (forma integrale),
  teorema di Gauss (forma differenziale),
  equazioni di Maxwell,
  equazione di Poisson,
  equazione di Laplace,
  induzione elettrostatica,
  tubo di linee di forza,
  teorema di Coulomb,
  schermo elettrostatico,
  gabbia di Faraday,
  pozzo di Faraday,
  capacit\'a del conduttore,
  condensatore,
  armature del condensatore,
  capacit\'a del condensatore,
  condensatore piano,
  condensatore cilindrico,
  condensatori in serie/parallelo,
  coefficienti di capacit\'a,
  coefficienti di potenziale,
  elettrometro,
  elettrometro condensatore,
  problema di Dirichlet,
  metodo delle cariche immagini,
  metodo della separazione di variabili,
\end{multicols}



\end{document}
