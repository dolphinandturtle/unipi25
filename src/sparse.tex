\documentclass[8pt, reqno, hidelinks]{amsart}
%\documentclass[8ptpt]{report}
\usepackage[margin=1.0cm]{geometry}
\usepackage[utf8]{inputenc}
\usepackage{multicol, wrapfig, blindtext, booktabs}
\usepackage{amsmath, amsthm, amsfonts, amssymb, mathrsfs, physics}
\usepackage{stackrel}
\usepackage{dsfont}
\usepackage{graphicx}
\usepackage{booktabs}
\usepackage{minted}
\usepackage{hyperref, bookmark}
\usepackage{cancel}
\usepackage{extarrows}
\usepackage{pagecolor}
\usepackage{tikz}
\usepackage{tikz-cd}
\usepackage{svg}
\usepackage{subcaption}
\usepackage{mathtools}
\usepackage{centernot}
\usepackage{xcolor}
\DeclarePairedDelimiter{\ceil}{\lceil}{\rceil}
\newenvironment{Figure}
  {\par\medskip\noindent\minipage{\linewidth}}
  {\endminipage\par\medskip}

% Standard Sets
\newcommand{\naturals}{\mathbb{N}}
\newcommand{\whole}{\mathbb{Z}}
\newcommand{\rationals}{\mathbb{Q}}
\newcommand{\reals}{\mathbb{R}}
\newcommand{\complex}{\mathbb{C}}
\newcommand{\field}{\mathbb{K}}
\newcommand{\cm}{\text{CM}}
\newcommand{\due}[1]{{^\star}#1}

\definecolor{blue}{HTML}{222255} %#0044cc
\definecolor{red}{HTML}{552222} %#ff3333
\definecolor{light-green}{HTML}{cdff99}
\definecolor{light-purple}{HTML}{ffcccc}
\definecolor{light-orange}{HTML}{ffeec0} %#ffdd99



%\newcommand{\mychapter}[1]{\section*{\texorpdfstring{\colorbox{light-purple}{#1}}{#1}}}
\newcommand{\mychapter}[1]{\section*{\texorpdfstring{\fbox{#1}}{#1}}}
\newcommand{\mysection}[1]{\section*{\texorpdfstring{\colorbox{light-orange}{#1}}{#1}}}
\newcommand{\mydefinition}[1]{\subsection*{\textcolor{red}{#1}}}
\newcommand{\mytheorem}[1]{\subsection*{\textcolor{blue}{#1}}}
\newcommand{\myparagraph}[1]{\paragraph{\textbf{#1}}}
\newcommand{\myproof}[1]{\tiny{\textcolor{gray}{#1}}}



\begin{document}

\begin{multicols}{3}
  \mydefinition{(Homo)morphism}
  A structure-preserving map between two algebraic structures of the same type.

  \mydefinition{Isomorphism}
  A \textit{homomorphism} that can be reversed by an inverse mapping.

  \mydefinition{Endomorphism}
  A \textit{homomorphism} from an object to itself.

  \mydefinition{Automorphism}
  A homomorphism from an object to itself that can be reversed by an inverse mapping.
  (\textit{isomorphism} + \textit{endomorphism})

  \mydefinition{Local and Non-local matrix operations}
  In matter of matrices, local operations are those who act on one entry, if not than non-local.

  \mydefinition{Convex Hull}
  The set of lines tangent to a function;
  \begin{equation}
    \label{eq:convex-hull}
    (m, q) \in CH(f) \iff \exists!x : f(x) = mx + q
  \end{equation}
  from which it is always possible to reconstruct the function,
  \begin{equation}
    f(x) = \max_{m, q}\{mx + q\}
  \end{equation}
  furthermore if the function is \textit{convex} this set can be
  represented as a function itself
  \begin{equation}
    f(x) = \max_{m}\{mx + q(m)\}
  \end{equation}

  \mydefinition{Legendre Transform}
  The \underline{opposite} of a function's \textit{convex hull}.
  \begin{equation}
    \label{eq:legendre-transform}
    f^*(x^*):= -q(m) = \max_{x}\{mx - f(x)\}
  \end{equation}

\end{multicols}


\mychapter{Sources}
\begin{enumerate}
\item \url{https://en.wikipedia.org/wiki/Endomorphism}
\end{enumerate}

\end{document}
