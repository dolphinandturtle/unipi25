\documentclass[8pt, reqno, hidelinks]{amsart}
%\documentclass[8ptpt]{report}
\usepackage[margin=1.0cm]{geometry}
\usepackage[utf8]{inputenc}
\usepackage{multicol, wrapfig, blindtext, booktabs}
\usepackage{amsmath, amsthm, amsfonts, amssymb, mathrsfs, physics}
\usepackage{stackrel}
\usepackage{dsfont}
\usepackage{graphicx}
\usepackage{booktabs}
\usepackage{minted}
\usepackage{hyperref, bookmark}
\usepackage{cancel}
\usepackage{extarrows}
\usepackage{pagecolor}
\usepackage{tikz}
\usepackage{tikz-cd}
\usepackage{svg}
\usepackage{subcaption}
\usepackage{mathtools}
\usepackage{centernot}
\usepackage{xcolor}
\DeclarePairedDelimiter{\ceil}{\lceil}{\rceil}
\newenvironment{Figure}
  {\par\medskip\noindent\minipage{\linewidth}}
  {\endminipage\par\medskip}

% Standard Sets
\newcommand{\naturals}{\mathbb{N}}
\newcommand{\whole}{\mathbb{Z}}
\newcommand{\rationals}{\mathbb{Q}}
\newcommand{\reals}{\mathbb{R}}
\newcommand{\complex}{\mathbb{C}}
\newcommand{\field}{\mathbb{K}}
\newcommand{\cm}{\text{CM}}
\newcommand{\due}[1]{{^\star}#1}

\definecolor{blue}{HTML}{222255} %#0044cc
\definecolor{red}{HTML}{552222} %#ff3333
\definecolor{light-green}{HTML}{cdff99}
\definecolor{light-purple}{HTML}{ffcccc}
\definecolor{light-orange}{HTML}{ffeec0} %#ffdd99



%\newcommand{\mychapter}[1]{\section*{\texorpdfstring{\colorbox{light-purple}{#1}}{#1}}}
\newcommand{\mychapter}[1]{\section*{\texorpdfstring{\fbox{#1}}{#1}}}
\newcommand{\mysection}[1]{\section*{\texorpdfstring{\colorbox{light-orange}{#1}}{#1}}}
\newcommand{\mydefinition}[1]{\subsection*{\textcolor{red}{#1}}}
\newcommand{\mytheorem}[1]{\subsection*{\textcolor{blue}{#1}}}
\newcommand{\myparagraph}[1]{\paragraph{\textbf{#1}}}
\newcommand{\myproof}[1]{\tiny{\textcolor{gray}{#1}}}

\usepackage[english]{babel}


\begin{document}

\mychapter{Keywords}
\begin{multicols}{3}
  Discreteness,
  diffraction,
  coherence,
  linear vector space,
  discrete vectors,
  spaces of functions,
  linearly independent,
  linearly dependent,
  dimension of the space,
  basis for the space,
  inner product,
  antilinear,
  orthogonal,
  norm (or lenght),
  Schwwarz's inequality,
  the triangle inequality,
  orthonormal,
  dual space,
  linear functionals,
  Riesz theorem,
  Dirac's bra and ket notation,
  linear operator,
  operator equality,
  operator associativity,
  operator commutativity,
  outer product,
  self-adjoint,
  Hermitian,
  eigenvector,
  eigenvalue,
  degenerator eigenvectors,
  orthonormal set of eigenvectors,
  complete operators,
  function of an operator,
  projection operator,
  Stieltjes integral,
  measure,
  spectral theorem,
  Hilbert space,
  commuting sets of operators,
  convergent sum of vectors,
  conjugate space,
  nuclear space,
  rigged Hilbert space,
  generalized spectral theorem,
  extended space,
  probability operators: negation, conjunction and disjunction,
  probability axioms,
  mutually exclusive,
  addition of probabilities for exclusive events,
  Baye's theorem,
  principle of inverse probability,
  independence of an event,
  statistical (or stochastic) independence,
  interpretations of probability:
  limit frequency interpretation, propensity interpretation,
  inductive inference, propositions, objective and subjective interpretations,
  binomail probability distribution,
  Chebyshev's inequality,
  theory of inductive inference,
  most probable value,
  first postulate,
  dynamical variable,
  linear operator,
  particle preparation and measurement,
  state,
  observable,
  state of preparation procedure,
  state-ensemble association,
  second postulate,
  state operator,
  statistical operator,
  density matrix,
  ensemble interpretation criticism,
  non-negative operator,
  strong second postulate,
  strong first postulate,
  general states,
  pure states,
  convex set,
  convex combinations,
  pure state operator,
  state vector,
  non-pure state,
  probability distributions,
  discrete spectrum,
  eigenstates,
  continuous spectrum,
  state function,
  filter function,
  correspondence rules,
  Wigner theorem,
  infinitesimal unitary transformations,
  generator of the family of unitary operators,
  symmetries of space-time,
  Galilei group,
  exponential Hermitian generator,
  pure displacements,
  rotation transformation,
  multiples of identity,
  anti-symmetric commutators,
  Jacobi identity,
  commutation relations for the generators of the Galilei group of transformations,
  identification of operators with dynamical variables,
  free particle,
  position operator,
  velocity operator,
  space displacement,
  displacement in velocity space,
  symmetry generators,
  irreducible set,
  spin,
  Hamiltonian,
  scalar potential,
  vector potential,
  mass,
  momentum,
  angular momentum,
  energy,
  Kronecker product,
  knematically independent,
  joint probability distribution,
  statistical independence,
  uncorrelated state,
  Poisson bracket,
  quantizing the classical system,
  time (in)dependent Hermitian,
  Schr\"odinger picture,
  Heisenber picture,
  Schr\"odinger operator,
  Heisenber operator,
  constant of motion,
  stationary state,
  representation,
  coordinate representation,
  position operator,
  momentum operator,
  Schr\"odinger wave equation,
  probability flux,
  singular point,
  physically realizable states,
  transmission coefficient,
  vacuum tunneling,
  work function,
  scanning tunneling microscope,
  path integral,
  propagator,
  action,
  imaginary time,
  canonical ensemble  
\end{multicols}

\begin{multicols}{3}
  \mysection{Oscillatore armonico}
  Scriviamo l'hamlitoniana
  \begin{equation*}
    \hat{H} = \frac{\hat{P}^2}{2m} + \frac{1}{2}m\omega^2\hat{X}^2
  \end{equation*}
  \'e conveniente riscalare gli operatori con le ``lunghezza del problema''
  \begin{equation}
    \label{eq:operatori-riscalati}
    \begin{gathered}
      % Come mai abbiamo riscalato in questa particolare maniera?
      \hat{x} = \hat{X} / l_{\omega} ~, \quad l_{\omega} = \sqrt{\frac{\hbar}{m\omega}} \\
      \hat{p} = \hat{P} / p_{\omega} ~, \quad p_{\omega} = \sqrt{m\omega\hbar}
    \end{gathered}
  \end{equation}
  se sostituiamo le quantit\'a all'interno dell'hamiltoniana
  \begin{equation}
    \label{eq:hamiltoniana-riscalata}
    \begin{gathered}
      \hat{H} = E_{\omega}(\hat{p}^2 + \hat{x}^2) ~, \quad E_{\omega} = \frac{\hbar\omega}{2} \\
      \left[\hat{X}, \hat{P}\right] = i\hbar \to \left[\hat{x}, \hat{p}\right] = i
    \end{gathered}
  \end{equation}

  \begin{equation}
    \begin{gathered}
      % Perch\'e abbiamo definito questo operatore?
      \hat{a} = \frac{1}{\sqrt{2}}\left(\hat{x} + i\hat{p}\right)
      ~, \quad
      \hat{a}^{\dagger} = \frac{1}{\sqrt{2}}\left(\hat{x} - i\hat{p}\right) \\
      \hat{x} = \frac{1}{\sqrt{2}}(a + a^{\dagger})
      ~, \quad
      \hat{p} = \frac{1}{\sqrt{2}}(a - a^{\dagger})\\
      [\hat{a}, \hat{a}^{\dagger}] = 1
    \end{gathered}
  \end{equation}

  Riscriviamo l'hamiltoniana in termini dei nuovi operatori
  \begin{align}
    \begin{split}
      \hat{H} &= E_{\omega}\left(a^{\dagger}a + aa^{\dagger}\right) \\
              &= 2E_{\omega}\left(a^{\dagger}a + \frac{1}{2}\right) \\
              &= \hbar\omega\left(\hat{N} + \frac{1}{2}\right)
                ~, \quad \hat{N} \equiv a^{\dagger}a
    \end{split}
  \end{align}

  tenere a mente
  \begin{equation}
    \label{eq:commutatori-a}
    \left[\hat{N}, \hat{a}\right] = -\hat{a}
    ~, \quad
    \left[\hat{N}, \hat{a}^{\dagger}\right] = \hat{a}
  \end{equation}

  Soluzione:
  \begin{equation}
    \hat{N}\ket{\nu} = \nu\ket{\nu}
  \end{equation}

  \begin{enumerate}
  \item Mostriamo che $\nu > 0$
    \begin{equation}
      \begin{gathered}
        a\ket{\nu} \geq 0
        ~\wedge~
        \left[\bra{\nu}a^{\dagger}a\ket{\nu}
        = \bra{\nu}\hat{N}\ket{\nu} = \nu\bra{\nu}\ket{\nu}\right] \\
        \implies \nu \geq 0
      \end{gathered}
    \end{equation}

  \item Cosa?
    \begin{equation}
      \begin{gathered}
        \hat{N} \hat{a} \ket{\nu} = \hat{a}\nu\ket{\nu} \\
        (\ref{eq:commutatori-a}) \implies \hat{N}\hat{a} = \hat{a}\hat{N} - \hat{a} \\
        \hat{N}\hat{a}\ket{\nu} = \hat{a}\hat{N} - \hat{a}\ket{\nu} = (\nu - 1) \hat{a}\ket{\nu}
      \end{gathered}
    \end{equation}

  \item C'\'e una contraddizione che pu\'o solo essere risolta da una condizione aggiuntiva
    \begin{equation}
      a\ket{0} = 0
    \end{equation}

  \item Dunque lo stato fondamentale
    \begin{equation}
      \hat{N}\ket{0} = 0
    \end{equation}

  \item Gli autovalori devono essere interi.

  \item Lo spettro di $\hat{N}$ \'e dato dall'insieme dei numeri naturali.
    \begin{equation}
      \begin{gathered}
        Na^\dagger\ket{\nu} = (a^\dagger + a^\dagger)\ket{\nu} = (\nu + 1)a^\dagger\ket{\nu} \\
        \hat{H} = \hbar\omega\left(\hat{N}+\frac{1}{2}\right) = \hbar\omega\left(n+\frac{1}{2}\right)
      \end{gathered}
    \end{equation}
  \end{enumerate}

  \begin{equation}
    \hat{H} = \hbar\omega\left(\hat{N} + \frac{1}{2}\right)
    \to
    E_n = \hbar\omega\left(n + \frac{1}{2}\right)
  \end{equation}

  \begin{equation}
    a^+\ket{n} = \sqrt{n+1}\ket{n+1}
  \end{equation}

\end{multicols}

\end{document}
