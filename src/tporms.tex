\documentclass[8pt, reqno, hidelinks]{amsart}
%\documentclass[8ptpt]{report}
\usepackage[margin=1.0cm]{geometry}
\usepackage[utf8]{inputenc}
\usepackage{multicol, wrapfig, blindtext, booktabs}
\usepackage{amsmath, amsthm, amsfonts, amssymb, mathrsfs, physics}
\usepackage{stackrel}
\usepackage{dsfont}
\usepackage{graphicx}
\usepackage{booktabs}
\usepackage{minted}
\usepackage{hyperref, bookmark}
\usepackage{cancel}
\usepackage{extarrows}
\usepackage{pagecolor}
\usepackage{tikz}
\usepackage{tikz-cd}
\usepackage{svg}
\usepackage{subcaption}
\usepackage{mathtools}
\usepackage{centernot}
\usepackage{xcolor}
\DeclarePairedDelimiter{\ceil}{\lceil}{\rceil}
\newenvironment{Figure}
  {\par\medskip\noindent\minipage{\linewidth}}
  {\endminipage\par\medskip}

% Standard Sets
\newcommand{\naturals}{\mathbb{N}}
\newcommand{\whole}{\mathbb{Z}}
\newcommand{\rationals}{\mathbb{Q}}
\newcommand{\reals}{\mathbb{R}}
\newcommand{\complex}{\mathbb{C}}
\newcommand{\field}{\mathbb{K}}
\newcommand{\cm}{\text{CM}}
\newcommand{\due}[1]{{^\star}#1}

\definecolor{blue}{HTML}{0044cc}
\definecolor{red}{HTML}{ff3333}
\definecolor{light-green}{HTML}{cdff99}
\definecolor{light-purple}{HTML}{ffcccc}
\definecolor{light-orange}{HTML}{ffdd99}



%\newcommand{\mychapter}[1]{\section*{\texorpdfstring{\colorbox{light-purple}{#1}}{#1}}}
\newcommand{\mychapter}[1]{\section*{\texorpdfstring{\fbox{#1}}{#1}}}
\newcommand{\mysection}[1]{\section*{\texorpdfstring{\colorbox{light-orange}{#1}}{#1}}}
\newcommand{\mydefinition}[1]{\subsection*{\textcolor{red}{#1}}}
\newcommand{\mytheorem}[1]{\subsection*{\textcolor{blue}{#1}}}
\newcommand{\myparagraph}[1]{\paragraph{\textbf{#1}}}
\newcommand{\myproof}[1]{\tiny{\textcolor{gray}{#1}}}

\usepackage[english]{babel}

\title{The principle of relativity made simple}
\author{Alessandro Rayan Ahmed}
\date{\today}

\begin{document}

\begin{multicols}{3}
  \maketitle
  \mysection{Special principle of relativity}
  The physical laws that hold in a system are invariant in form
  in relation to any other system moving in uniform translation.

  \begin{equation}
    \label{eq:v-isotropy}
    \begin{gathered}
      \lambda R_{\vec{v}} = R_{\vec{v}}\lambda \\
      \lambda_{\vec{u}}(-\mathds{1}_{\vec{v}}) = (-\mathds{1}_{\vec{v}})\lambda_{\vec{v}} \\
    \end{gathered}
  \end{equation}

  %\begin{equation}
  %  \label{eq:x-decomposed}
  %  \begin{gathered}
  %    \bar{X} = \bar{X}_{\parallel \bar{V}} + \bar{X}_{\parallel \bar{T}} - \bar{X}_{\parallel\bar{V}\parallel\bar{T}} + \bar{X}_{\perp\bar{V}\perp\bar{T}} \\
  %    \begin{cases}
  %      \bar{X}_{\parallel \bar{V}} = \left(\bar{X}*\bar{V}\right)\bar{V} \\
  %      \bar{X}_{\parallel \bar{T}} = \left(\bar{X}*\bar{T}\right)\bar{T} \\
  %      \bar{X}_{\parallel\bar{V}\parallel\bar{T}} = \left(\left(\left(\bar{X}*\bar{V}\right)\bar{V}\right)*\bar{T}\right)\bar{T}
  %    \end{cases}
  %  \end{gathered}
  %\end{equation}
  %
  %\begin{align*}
  %  \begin{split}
  %    \bar{X} &= \left(\bar{X}*\bar{V}\right)\bar{V} + \left(\bar{X} - \left(\bar{X}*\bar{V}\right)\bar{V}\right) \\
  %            &\equiv \bar{X}_{\parallel \bar{V}} + \bar{X}_{\perp \bar{V}}
  %  \end{split}
  %\end{align*}
  %\begin{align*}
  %  \begin{split}
  %    \bar{X}_{\perp\bar{V}} &= \left(\bar{X}_{\perp\bar{V}}*\bar{T}\right)\bar{T} + \left(\bar{X}_{\perp\bar{V}} - \left(\bar{X}_{\perp\bar{V}}*\bar{T}\right)\bar{T}\right)\\
  %                           &\equiv \left(\left(\bar{X} - \left(\bar{X}*\bar{V}\right)\bar{V}\right)*\bar{T}\right)\bar{T} + \bar{X}_{\perp\bar{V}\perp\bar{T}} \\
  %                           &\equiv \bar{X}_{\parallel\bar{T}} - \bar{X}_{\parallel\bar{V}\parallel\bar{T}} + \bar{X}_{\perp\bar{V}\perp\bar{T}}
  %  \end{split}
  %\end{align*}

  The rotation is substituted
  \begin{equation}
    \label{eq:-constraint}
    \left[\lambda, \left(\mathds{1} - \bar{P}^2\right)\left(\bar{V}^2 + \bar{T}^2 - \bar{V}^2\bar{T}^2\right) + \bar{P}^2\right] = \bar{0}
  \end{equation}
  
  \begin{equation}
    \label{eq:flipping-constraint}
    \left(\lambda_{\vec{u}} - \lambda_{\vec{v}}^t\right)\left(\mathds{1} - 2\bar{V}^2\right) = \bar{0} 
  \end{equation}
    
  %\begin{align*}
  %  \begin{split}
  %    &\lambda R_{\vec{v}}\bar{X} = \lambda\bar{X}_{\parallel\vec{v}} + \lambda R_{\vec{v}}\bar{X}_{\perp\vec{v}} \\
  %    &= \lambda\circ\left(\bar{V}^2 + \bar{T}^2 - \bar{V}^2\bar{T}^2\right)\bar{X} + \lambda R_{\vec{v}}\bar{X}_{\perp\bar{V}\perp\bar{T}}
  %  \end{split}
  %\end{align*}
  %
  %\begin{align*}
  %  \begin{split}
  %    &R_{\vec{v}}\lambda\bar{X} = \left(\lambda\bar{X}\right)_{\parallel\vec{v}} + R_{\vec{v}}\left(\lambda\bar{X}\right)_{\perp\vec{v}} \\
  %    &= \left(\bar{V}^2 + \bar{T}^2 - \bar{V}^2\bar{T}^2\right)\lambda\bar{X} + R_{\vec{v}}\lambda\bar{X}_{\perp\bar{V}\perp\bar{T}}
  %  \end{split}
  %\end{align*}
  %
  %\begin{equation*}
  %  \left[\lambda, \left(\bar{V}^2 + \bar{T}^2 -\bar{V}^2\bar{T}^2\right)\bar{X} + R_{\vec{v}}\bar{X}_{\perp\bar{V}\perp\bar{T}}\right] = \bar{0}
  %\end{equation*}

\end{multicols}

\end{document}
