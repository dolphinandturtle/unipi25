\documentclass[12pt]{report}
\usepackage{tikz}
\usepackage{circuitikz}
\usepackage{amsmath}


\title{Circuito ganzo}

\begin{document}

\maketitle

\chapter*{Caso finito}
Nelle schematiche ogni resistenza \'e di valore uguale noto $R$
\begin{figure}[!ht]
\centering
\resizebox{0.6\textwidth}{!}{%
\begin{circuitikz}
\tikzstyle{every node}=[font=\LARGE]
\draw (4.75,15) to[short, -o] (4.25,15) ;
\draw (4.75,13) to[short, -o] (4.25,13) ;
\draw (4.75,15) to[R] (6.75,15);
\draw (4.75,13) to[R] (6.75,13);
\draw (6.75,15) to[R] (6.75,13);
\draw (6.75,15) to[R] (9.5,15);
\draw (6.75,13) to[R] (9.5,13);
\draw (9.5,15) to[R] (9.5,13);
\end{circuitikz}
}%

\label{fig:my_label}
\end{figure}

Un modo semplice di risolvere il circuito \'e quello di rappresentarlo in
maniera diversa

\begin{figure}[!ht]
\centering
\resizebox{0.8\textwidth}{!}{%
\begin{circuitikz}
\tikzstyle{every node}=[font=\LARGE]
\draw (4,12.25) to[short, -o] (3.75,12.25) ;
\draw (4,12.25) to[R] (6.5,12.25);
\draw (6.5,13.25) to[short] (6.5,11.25);
\draw (6.5,13.25) to[R] (8.25,13.25);
\draw (8.25,13.25) to[R] (9.25,13.25);
\draw (9.25,13.25) to[R] (11,13.25);
\draw (6.5,11.25) to[R] (11,11.25);
\draw (11,13.25) to[short] (11,11.25);
\draw (11,12.25) to[R] (13.25,12.25);
\draw (13.25,12.25) to[short, -o] (13.75,12.25) ;
\end{circuitikz}
}%

\label{fig:my_label}
\end{figure}

\begin{equation}
  \label{sol:finito}
  R_{eq} = R + \left(\frac{1}{R} + \frac{1}{3R}\right)^{-1} + R
\end{equation}

\chapter*{Caso infinito}
\begin{figure}[!ht]
\centering
\resizebox{1\textwidth}{!}{%
\begin{circuitikz}
\tikzstyle{every node}=[font=\LARGE]
\draw (4.75,15) to[short, -o] (4.25,15) ;
\draw (4.75,13) to[short, -o] (4.25,13) ;
\draw (4.75,15) to[R] (6.75,15);
\draw (4.75,13) to[R] (6.75,13);
\draw (6.75,15) to[R] (6.75,13);
\draw (6.75,15) to[R] (9.5,15);
\draw (6.75,13) to[R] (9.5,13);
\draw (9.5,15) to[R] (9.5,13);
\draw (9.5,15) to[R] (12.25,15);
\draw (9.5,13) to[R] (12.25,13);
\draw [dashed] (12.25,15) -- (12.25,13);
\end{circuitikz}
}%

\label{fig:my_label}
\end{figure}

\begin{figure}[!ht]
\centering
\resizebox{1\textwidth}{!}{%
\begin{circuitikz}
\tikzstyle{every node}=[font=\LARGE]
\draw (4,12.25) to[short, -o] (3.75,12.25) ;
\draw (13.25,12.25) to[short, -o] (13.75,12.25) ;
\draw (3.75,12.25) to[short] (3.75,9.75);
\draw (13.75,12.25) to[short] (13.75,9.75);
\draw (3.75,9.75) to[R] (13.75,9.75);
\draw (1,11) to[short, -o] (0.75,11) ;
\draw (1,11) to[R] (3.75,11);
\draw (13.75,11) to[R] (16.25,11);
\draw (16.25,11) to[short, -o] (16.5,11) ;
\draw (0.75,11) to[short] (0.75,8);
\draw (16.5,11) to[short] (16.5,8);
\draw (0.75,8) to[R] (16.5,8);
\draw (-1.75,9.5) to[short, -o] (-2,9.5) ;
\draw (-1.75,9.5) to[R] (0.75,9.5);
\draw (16.5,9.5) to[R] (18.75,9.5);
\draw (18.75,9.5) to[short, -o] (19.5,9.5) ;
\draw [dashed] (6.5,13.25) -- (6.5,11);
\draw [dashed] (6.5,13.25) -- (11,13.25);
\draw [dashed] (6.5,11) -- (11,11);
\draw [dashed] (4,12.25) -- (6.5,12.25);
\draw [dashed] (11,13.25) -- (11,11);
\draw [dashed] (11,12.25) -- (13.25,12.25);
\end{circuitikz}
}%

\label{fig:my_label}
\end{figure}


\begin{equation}
  \label{sol:infinito}
  R_{eq, n+1} = R + \left(\frac{1}{R} + \frac{1}{R_{eq, n}}\right)^{-1} + R    
\end{equation}

Senza particolari accertezze matematiche \'e ragionevole per $n\to\infty$

\begin{equation}
  \label{sol:infinito}
  \begin{gathered}
    R_{eq} = R + \left(\frac{1}{R} + \frac{1}{R_{eq}}\right)^{-1} + R
    \implies
    (R_{eq} - 2R)(R+R_{eq}) = RR_{eq} \\
    R_{eq} = (1 + \sqrt{3})R
  \end{gathered}
\end{equation}

\end{document}