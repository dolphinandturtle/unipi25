\documentclass[8pt, reqno, hidelinks]{amsart}
%\documentclass[8ptpt]{report}
\usepackage[margin=1.0cm]{geometry}
\usepackage[utf8]{inputenc}
\usepackage{multicol, wrapfig, blindtext, booktabs}
\usepackage{amsmath, amsthm, amsfonts, amssymb, mathrsfs, physics}
\usepackage{stackrel}
\usepackage{dsfont}
\usepackage{graphicx}
\usepackage{booktabs}
\usepackage{minted}
\usepackage{hyperref, bookmark}
\usepackage{cancel}
\usepackage{extarrows}
\usepackage{pagecolor}
\usepackage{tikz}
\usepackage{tikz-cd}
\usepackage{svg}
\usepackage{subcaption}
\usepackage{mathtools}
\usepackage{centernot}
\usepackage{xcolor}
\DeclarePairedDelimiter{\ceil}{\lceil}{\rceil}
\newenvironment{Figure}
  {\par\medskip\noindent\minipage{\linewidth}}
  {\endminipage\par\medskip}

% Standard Sets
\newcommand{\naturals}{\mathbb{N}}
\newcommand{\whole}{\mathbb{Z}}
\newcommand{\rationals}{\mathbb{Q}}
\newcommand{\reals}{\mathbb{R}}
\newcommand{\complex}{\mathbb{C}}
\newcommand{\field}{\mathbb{K}}
\newcommand{\cm}{\text{CM}}
\newcommand{\due}[1]{{^\star}#1}

\definecolor{blue}{HTML}{0044cc}
\definecolor{red}{HTML}{ff3333}
\definecolor{light-green}{HTML}{cdff99}
\definecolor{light-purple}{HTML}{ffcccc}
\definecolor{light-orange}{HTML}{ffdd99}



%\newcommand{\mychapter}[1]{\section*{\texorpdfstring{\colorbox{light-purple}{#1}}{#1}}}
\newcommand{\mychapter}[1]{\section*{\texorpdfstring{\fbox{#1}}{#1}}}
\newcommand{\mysection}[1]{\section*{\texorpdfstring{\colorbox{light-orange}{#1}}{#1}}}
\newcommand{\mydefinition}[1]{\subsection*{\textcolor{red}{#1}}}
\newcommand{\mytheorem}[1]{\subsection*{\textcolor{blue}{#1}}}
\newcommand{\myparagraph}[1]{\paragraph{\textbf{#1}}}
\newcommand{\myproof}[1]{\tiny{\textcolor{gray}{#1}}}

\usepackage[english]{babel}

\begin{document}
\mychapter{Propositions}

\begin{multicols}{3}
  %\raggedcolumns
  \mysection{Vector spaces}

  % TODO
  \mydefinition{Complex numbers}
  Ordered pairs of real numbers with addition and multiplication consistent with
  the algebra of a sum between a number and a scaled \textit{root of the negative unit}.

  \mytheorem{Complex arithemtic}
  Commutativity, associativity, identities, inverses and product distributivity over addition.

  \mydefinition{Subtraction/Division}
  Adding/Multiplying by the \textit{inverse} of the latter operand.

  \mydefinition{List of lenght}
  A \textit{finite} set with an \textit{ordering}.

  \mydefinition{Coordinate space}
  \myparagraph{Definition}
  The set of \textit{scalar lists} of a given lenght.
  \myparagraph{Operations}
  Addition and scalar multiplication are defined \textit{coordinatewise}.
  \myparagraph{Elements}
  The \textit{identity element} is a list of zeros, the \textit{inverse}
  of an element is its coordinatewise \textit{scalar inverse}.

  \mydefinition{Vector operations}
  \myparagraph{Addition}
  A map to domain endomaps.
  \myparagraph{Scalar multiplication}
  A map from a \textit{field} to some endomaps.

  \mydefinition{Vector space}
  An \textit{abelian group} under addition with a \textit{ring homomorphism} from a \textit{field} into the group's \textit{endomorphism ring} as scalar multiplication.

  \mydefinition{Standard vector spaces}
  A real/complex vector space has scalar multiplication defined on the \textit{real/complex field}.

  \mytheorem{Vector arithemtic}
  Identity and inverses are unique. Null scaling and scaling null is null. Inverse identity scaling is inverse.

  \mysection{Subspaces}
  \mydefinition{Subspace}
  A subset of a vector space with its \textit{additive identity}, \textit{operations} and properties.
  \myparagraph{Characterization}
  If and only if it contains the additive identity and is \textit{closed} under the operations.

  \mydefinition{Sum of subspaces}
  The set of \textit{sums} between elements of respective subspaces.
  \myparagraph{Structure}
  The smallest \textit{subspace} containing the subspaces \textit{individually}.

  \mydefinition{Direct-sum}
  A sum of subspaces whose elements can be written as a \textit{unique} sum.
  \myparagraph{Characterization}
  If and only if its \textit{null vector} can be written as a \textit{unique} sum of null vectors.

  \mytheorem{Direct-sum of a pair}
  If and only if their \textit{intersection} is trivial.

  \mysection{Span and linear independence}
  \mydefinition{Linear combination}
  A \textit{sum over entrywise products} of scalars and vectors.
  
  \mydefinition{Span of vectors}
  The set of all the vector's \textit{linear combinations}.
  \myparagraph{Structure}
  The smallest subspace containing the vectors.

  \mydefinition{Polynomial with scalar coefficients}
  A field endo-map that sums over the entrywise products of scalars and increasing \textit{whole powers}.
  \myparagraph{Degree}
  The \textit{greatest power} taken, not scaled by zero.
  
  \mydefinition{Cardinality}
  A vector space spanned by some list of \textit{its} vectors
  is \textbf{finite-dimensional},
  if not then \textbf{infinite-dimensional}.
  
  \mydefinition{Linear dependence}
  A list of vectors whose null linear combination implies all coefficients to be zero is \textit{linearly independent},
  if not then \textit{linearly dependent}.
  \myparagraph{Characterization}
  Some in a linearly dependent list are in the span of their preceding, \textit{they don't affect the first span}.
  
  \mytheorem{List lenght constraint}
  The lenght of a linearly independent list \textit{cannot be greater} than that of a spanning list.
  
  \mytheorem{Subspace cardinality}
  A subspace of a finite-dimensional vector space is finite-dimensional.
  
  \mysection{Bases}
  \mydefinition{Basis of a vector space}
  A list of vectors that is linearly \textit{independent} and that \textit{spans} their vector space.
  \myparagraph{Characterization}
  If and only if every other vector can be written as a \textit{unique linear combination} of it.
  
  \mytheorem{List resizing}
  A spanning/independent list can be reduced/extended to a basis by removing/adding some elements.

  \mytheorem{Basis cardinality}
  Finite-dimensional vector spaces have a basis.
  
  \mytheorem{Subspace direct completion}
  For every subspace, in a finite-dimensional vector space there exists a \textit{complement} such that their direct sum equals the whole space.
  
  \mysection{Dimension}
  \mytheorem{Basis lenght lemma}
  Any pair of bases of a vector space have the \textit{same lenght}.
  
  \mydefinition{Dimension of a vector space}
  The \textit{lenght} of a basis of the vector space.
  
  \mytheorem{Dimension of a subspace}
  The dimension of a subspace \textit{cannot} be greater than that of the whole space.
  
  \mytheorem{Full dimension subspace}
  A subspace whose dimension is \textit{equal} to that of its whole space, is the whole space itself.
  
  \mytheorem{Basis lenght list}
  A list of vectors whose lenght is equal to the full dimension and is linearly independent or spanning, is also a basis.
  
  \mytheorem{Dimension of a sum}
  The dimension of the sum between a pair of subspaces is the sum of their dimensions minus the dimension of their intersection.

  \mysection{Linear maps}
  \mydefinition{Linear map}
  A \textit{group homomorphism} over addition that \textit{commutes} with scalar multiplication.
  \myparagraph{Characterization}
  Elements of a basis can be mapped everywhere, and the map is unique.
  \myparagraph{Operations}
  Pointwise \textit{addition} and \textit{scalar multiplication}.
  \myparagraph{Structure}
  Vector space.

  \mydefinition{Product of linear maps}
  The product of maps gives their \textit{composition}.

  \mytheorem{Map arithemtic}
  Associativity, identity and product distributivity over addition.

  \mytheorem{Null mapping}
  Null vectors map into null vectors.

  \mydefinition{Null space of a map}
  The set of elements mapped into \textit{null vectors}.
  \myparagraph{Structure}
  \textit{Subspace} of the domain.

  \mydefinition{Injective}
  Distinct action for distinct elements.
  \myparagraph{Characterization}
  If and only if the null space is trivial.

  \mydefinition{Range of a map}
  The set of mapped elements.
  \myparagraph{Structure}
  \textit{Subspace} of the \textit{codomain}.

  \mydefinition{Surjective}
  Whose \textit{range} equals its \textit{codomain}.

  \mytheorem{Fundamental theorem}
  A linear map's \textit{domain} dimension is equal to the sum between the dimensions of its \textit{null space} and \textit{range}.

  \mytheorem{Map classing}
  In matters of dimension, higher isn't \textit{into} lower and lower isn't \textit{onto} higher.

  \mytheorem{Solution of linear systems}
  \paragraph{\textbf{Homogeneous}}
  With more variables than equations and null costant terms, has \textit{non-zero solutions}.
  
  \paragraph{\textbf{Heterogeneous}}
  With more equation than variables has \textit{no solution} for some choice of constant terms.
  
  \mysection{Matrices}
  \mydefinition{Matrix}
  A \textit{discrete} rectangle of numbers with a \textit{row-column} index pair.
  \myparagraph{Operations}
  Elementwise \textit{addition} and \textit{scalar multiplication}.
  
  \mydefinition{Map matrix-form}
  Each column represents a \textit{mapped domain basis} as a linear combination of \textit{codomain basis} with column coefficients.

  \mydefinition{Matrix space}
  A set of matrices of \textit{fixed} geometry with matrix addition and scalar multiplication.
  \myparagraph{Structure}
  Vector space.
  \myparagraph{Dimension}
  The \textit{row-count} times the \textit{column-count}.
  
  \mydefinition{Matrix multiplication}
  \textit{Entry-wise} sum of the \textit{component-wise} product of row and column.
  
  \mytheorem{Matrix-form distributivity with maps}
  The matrix-form of a map distributes over \textit{addition}, \textit{scaling} and \textit{multiplication}.

  \mydefinition{Row/column matrices}
  A matrix with \textit{one row/column}.

  \mytheorem{Entry of multiplication}
  The product between the corresponding multiplicand row and multiplier column.

  \mytheorem{Column of multiplication}
  The product between the multiplicand matrix and the corresponding multiplier column.

  \mytheorem{Multiplication by column}
  A linear combination of the multiplicand columns with multiplier column coefficient.

  \mytheorem{Multiplication as a linear combination}
  Multiplication columns/rows can be written as the linear combinations of
  multiplicand/multiplier columns/rows with
  the corresponding multiplier/multiplicand column/row coefficients.
  
  \mydefinition{Column/row rank}
  Dimension of the span of the columns/rows of a matrix.

  \mydefinition{Transpose}
  A matrix automap that \textit{swaps} rows and columns.

  \mytheorem{Column-row factorization}
  Every matrix with a non trivial column-rank
  can be written as a product between
  a matrix of \textit{rank-width} and a matrix of \textit{rank-height}.

  \mytheorem{Row-column rank}
  The column rank of a matrix \textit{equals} its row rank.

  \mydefinition{Rank}
  The column rank of a matrix.

  \mysection{Isomorphisms}
  \mydefinition{Inverse}
  %A map whose product with its former commutes and gives the identity map.
  Given a map its inverse is such that their product commutes and gives the identity map.
  \myparagraph{Cardinality}
  Given a map the inverse is unique.
  \myparagraph{Characterization}
  If and only if it is injective and surjective.

  \mytheorem{Map class equivalence}
  \textit{Injectivity}, \textit{surjectivity} and \textit{invertibility} in a linear map between two spaces of equal \textit{finite-dimension} are all equivalent.

  \mytheorem{Semi-invertibility}
  If the product of \textit{finite-dimensional} linear maps is the identity, the product \textit{commutes}.

  \mydefinition{Isomorphism}
  An invertible linear map.
  \myparagraph{Existence}
  If two vector spaces have equal and finite dimension.

  \mytheorem{Matrix-form morphism}
  Given a basis of the vector spaces acted upon, the matrix form is an
  \textit{isomorphism} from the space of \textit{linear maps} to the \textit{space of matrices}.

  \mytheorem{Dimension of a linear map}
  A linear map's dimension is the product between the dimension of its \textit{domain} and \textit{codomain}.

  \mydefinition{Vector-matrix}
  A \textit{column-matrix} of coefficients from the linear combination of a basis representing the vector.

  \mytheorem{Matrix column explicitation}
  A column of a matrix-form is the vector-matrix of the corresponding mapped basis.
  
  \mytheorem{Maps as matrix multiplication}
  A linear mapping of a vector is equal to the matrix multiplication between their matrix-form respectively.
  
  \mytheorem{Range-rank}
  Given a \textit{finite-dimensional} linear map,
  the dimension of its \textit{range} is equal to
  the \textit{column rank} of its matrix form.
  
  \mydefinition{Identity matrix}
  Square matrix with left to right diagonal filled with 1's.

  \mydefinition{Inverse matrix}
  Given an \textit{invertible} matrix its inverse is such that their left/right product gives the \textit{idenity matrix}.

  \mytheorem{Matrix of map products}
  The matrix-form of a linear map product is the product of the linear maps' matrix-forms.
  
  \mytheorem{Identity inverse}
  The matrix of an identity map with respect to some bases is
  \textit{its} inverse with inverted bases.
  
  \mytheorem{Change-of-basis formula}
    
  \mytheorem{Matrix of inverse}
  The matrix of the inverse map is the inverse matrix of the map.

  \mysection{Products and Quotients}

  \mydefinition{Product of vector spaces}
  \myparagraph{Definition}
  The set of \textit{vector lists} from the vector spaces respectively.
  \myparagraph{Operations}
  Entrywise \textit{addition} and \textit{scalar multiplication}.
  \myparagraph{Structure} Vector space.
  \myparagraph{Dimension}
  The \textit{sum} of dimensions of the operand vector spaces.

  \mytheorem{Direct sum map}
  A linear map from product to sum that sums entrywise is
  \textit{into} if and only if the sum of spaces is \textit{direct}.

  \mytheorem{Direct sum dimension}
  Sums between vector spaces is direct if and only if its dimensions \textit{add up}.

  \mydefinition{Subset translate}
  A subset of a vector space \textit{summed elementwise} (translated) by one of its elements.

  \mytheorem{Equivalence class}
  Two translates of a subspace are \textit{equal} or \textit{disjoint}.

  \mydefinition{Quotient space}
  The set of all \textit{translates} of a vector space by elements of its subspace.
  \myparagraph{Operations}
  The \textit{sum} of translates is the translates by the sum, the \textit{scaled} translate
  is the translate by the scaled.
  \myparagraph{Structure} Vector space.
  \myparagraph{Dimension}
  The difference between the dimensions of the \textit{dividend} and \textit{divisor} respectively.

  \mydefinition{Quotient map}
  An \textit{induced} linear map from the domain modulo kernel to the codomain that
  acts the linear map on the \textit{translation vector}.
  
  \mytheorem{Kernel and range of tilde map}
  Quotient map into tilde map gives the map, tilde map is injective,
  tilde map's range is equal to that of its map and
  the quotient of the domain with the map kernel is isomorphic
  to its map.

  \mysection{Duality}

  \mydefinition{Linear functional}
  A linear map to a field.

  \mydefinition{Dual space}
  The set of linear functionals from a given space.
  \myparagraph{Dimension}
  The dimension of the domain.

  \mydefinition{Dual basis}
  The list of functionals that map corresponding basis entries
  into 1 and the rest into 0.

  \mytheorem{Coefficient extraction}
  A dual basis acted entrywise on a vector gives the coefficients
  to the latter's representation as a linear combination of the basis.

  \mytheorem{Dual basis coherence}
  The dual basis of a vector space is a basis of its dual.

  \mydefinition{Dual map}
  An induced map from the dual codomain to dual domain
  that lets the map get \textit{acted on}.

  \mytheorem{Algebraic properties of dual maps}
  Duality \textit{distributes} over addition and scalar multiplication.
  The dual of the product is the \textit{commuted product} of the duals.

  \mydefinition{Annihilator}
  The set of linear functionals that sends \textit{subdomain} elements to zero.
  \myparagraph{Structure} A \textit{subspace} of the \textit{dual domain}.

  \mytheorem{Subspace annihilator dimension}
  The difference between the dimensions of whole space and subspace respectively.

  \mytheorem{Subspace annihilator bounds}
  The annihilator is \textit{trivial} or the \textit{whole space} if and only if
  the subspace is the \textit{whole space} or is \textit{trivial} respectively.

  \mytheorem{Null space of a dual map}
  The annihilator of the map's range is the \textit{dual's kernel}.
  In matters of dimension the latter is the map's range, plus
  the codomain, and minus the domain.
  
  \mytheorem{Injective dual map}
  A finite-dimensional map is surjective if and only if its dual is injective.

  \mytheorem{Range of a dual map}
  The annihilator of the map's kernel is the \textit{dual's range}.
  The dimension of the map's range is equal to that of its dual's.

  \mytheorem{Surjective dual map}
  A finite-dimensional map is injective if and only if its dual is surjective.

  \mytheorem{Dual map matrix}
  A finite-dimensional map's matrix form traspose is the dual's matrix form.
  
  \mytheorem{Row-column rank}
  The column rank of a matrix \textit{equals} its row rank.

  %\mysection{Polynomials}
  %\mydefinition{Real/immaginary part}
  %\mydefinition{Complex conjugate}
  %\mydefinition{Absolute value}
  %\mytheorem{Properties of complex numbers}
  %\mydefinition{Zero of a polynomial}
  %\mytheorem{One-degree factoring}
  %Each zero of a polynomial corresponds to a degree-one factor.
  %\mytheorem{Count of zeros}
  %\mytheorem{Division algorithm}
  %\mytheorem{Fundamental theorem of algebra}
  %\myparagraph{First version}
  %\myparagraph{Second version}
  %\mytheorem{Complex pairs of zeros}
  %\mytheorem{Factorization of a quadratic polynomial}
  %\mytheorem{Factorization of a real polynomial}
  
  
\end{multicols}

\end{document}
