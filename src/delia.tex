\documentclass[8pt, reqno, hidelinks]{amsart}
%\documentclass[8ptpt]{report}
\usepackage[margin=1.0cm]{geometry}
\usepackage[utf8]{inputenc}
\usepackage{multicol, wrapfig, blindtext, booktabs}
\usepackage{amsmath, amsthm, amsfonts, amssymb, mathrsfs, physics}
\usepackage{stackrel}
\usepackage{dsfont}
\usepackage{graphicx}
\usepackage{booktabs}
\usepackage{minted}
\usepackage{hyperref, bookmark}
\usepackage{cancel}
\usepackage{extarrows}
\usepackage{pagecolor}
\usepackage{tikz}
\usepackage{tikz-cd}
\usepackage{svg}
\usepackage{subcaption}
\usepackage{mathtools}
\usepackage{centernot}
\usepackage{xcolor}
\DeclarePairedDelimiter{\ceil}{\lceil}{\rceil}
\newenvironment{Figure}
  {\par\medskip\noindent\minipage{\linewidth}}
  {\endminipage\par\medskip}

% Standard Sets
\newcommand{\naturals}{\mathbb{N}}
\newcommand{\whole}{\mathbb{Z}}
\newcommand{\rationals}{\mathbb{Q}}
\newcommand{\reals}{\mathbb{R}}
\newcommand{\complex}{\mathbb{C}}
\newcommand{\field}{\mathbb{K}}
\newcommand{\cm}{\text{CM}}
\newcommand{\due}[1]{{^\star}#1}

\definecolor{blue}{HTML}{0044cc}
\definecolor{red}{HTML}{ff3333}
\definecolor{light-green}{HTML}{cdff99}
\definecolor{light-purple}{HTML}{ffcccc}
\definecolor{light-orange}{HTML}{ffdd99}



%\newcommand{\mychapter}[1]{\section*{\texorpdfstring{\colorbox{light-purple}{#1}}{#1}}}
\newcommand{\mychapter}[1]{\section*{\texorpdfstring{\fbox{#1}}{#1}}}
\newcommand{\mysection}[1]{\section*{\texorpdfstring{\colorbox{light-orange}{#1}}{#1}}}
\newcommand{\mydefinition}[1]{\subsection*{\textcolor{red}{#1}}}
\newcommand{\mytheorem}[1]{\subsection*{\textcolor{blue}{#1}}}
\newcommand{\myparagraph}[1]{\paragraph{\textbf{#1}}}
\newcommand{\myproof}[1]{\tiny{\textcolor{gray}{#1}}}

\usepackage[italian]{babel}


\begin{document}

\section{``Lezioni di meccanica classica'' by Massimo D'Elia}
\mychapter{Kewords}
\begin{multicols}{3}
teorema di Noether,
hamiltoniana, trasformazioni/simmetrie/invarianza di gauge,
condizioni di Helmoltz, isocronismo pendolo, pendolo di Foucault (magnetico),
teorema di Bertrand, lagrangiana dell'oscillatore armonico smorzato, variet\'a,
gruppi (non)abeliani, gruppi di Lie, gruppo delle isometrie, trasformazioni
affini/ortogonali, gruppo ortogonale(-speciale), generatore infinitesimo,
spazio tangente, rappresentazione esponenziale del gruppo, teorema di Eulero
sulle rotazioni, formula di Baker-Campbell-Hausdorff, algebra di Lie, costanti
di struttura del gruppo, gruppo di Lorentz, tensori, rappresentazione
fondamentale, rappresentazione fedele/(ir)riducibile, covariante, angoli di
Eulero, angolo di precessione/nutazione, angolo di rotazione propria, tensore
d'inerzia, momenti principali d'inerzia, assi principali d'inerzia, trottola
assimetria/simmetrica/sferica, teoremi di K\"onig, teorema di Huygens-Steiner,
modi normali, matrice Hessiana, oscillatore armonico anisotropo, frequenza
degenere, pendolo sferico doppio, modello di Debye, corpo nero, condizioni
periodiche al contorno, risonanza, equazioni d'onda, equazione di Klein-Gordon,
d'Alambertiano,
equazione di sine-Gordon, solitoni, calcolo delle variazioni, sforzo
variazionale, principio di minima azione, equazioni di Eulero-Lagrange, lemma
fondamentale di calcolo delle variazioni, derivata funzionale, delta di Dirac,
Brachistocrona, geodetiche, soluzione di Johann Bernoulli, problema di Didone,
equazioni di Lagrange di $I^o$ tipo, moltiplicatore di Lagrange, catenaria,
principio di Fermat/Maupertuis, integrale funzionale, bolla di sapone
variazionale, problema di Plateau, spazio degli atti in moto, spazio delle
fasi, trasformata di Legendre, equazioni di Hamilton, funzionen di Routh, campo
e flusso Hamiltoniano, traiettorie di librazione/circolazione, parentesi di
Poisson, identit\'a di Jacobi,
teorema di Poisson, vettore di Laplace-Runge-Lenz, teorema di Liouville-Arnold,
sistemi super-integrabili, teorema di Liouville, equazione di continuit\'a,
equazione di Liouville, sistemi ergodici, invariante adiabatico, regola di
quantizzazione di Bohr-Sommerfeld, trasformazioni canoniche, trasformazioni puntuali, spazio
co-tangente, gruppo simplettico, equivalenza fra canonicit\'a e
simpletticit\'a, invarianza parentesi di Poisson, funzione generatrice e
criterio di esistenza, 1-forma, 2-forma, trasformazioni lineari simplettiche,
variabili azione-angolo, teoria delle perturbazioni, coordinate paraboliche,
metodi di Runge-Kutta, integratori simplettici, algoritmo leapfrog,
hamiltoniana ombra, funzione principale di Hamilton, l'equazione di
Hamilton-Jacobi, radice quantistica della meccanica classica, integrali
completi, velocit\'a di gruppo, principio della fase, stazionaria, funzione
d'onda, approssimazione semi-classica.
\end{multicols}

\mychapter{Condensed}
\begin{multicols}{3}
  %%%%%%%%%%%%%%%%%%%%%%%%%%%%%%%%%%%%%%%%%%%%%%%%%%%%%%%%%%%%%%%%%%%%%%%%%%%%%
  19)
  \href{https://it.wikipedia.org/wiki/Punto_materiale}
  {Punto materiale.}
  \href{https://it.wikipedia.org/wiki/Principi_della_dinamica}
  {Principi della dinamica.}
  \href{https://it.wikipedia.org/wiki/Sistema_di_riferimento_inerziale}
  {Sistemi di riferimento inerziali.}
  %%%%%%%%%%%%%%%%%%%%%%%%%%%%%%%%%%%%%%%%%%%%%%%%%%%%%%%%%%%%%%%%%%%%%%%%%%%%%
  20) Trasformazioni di Galileo. Massa inerziale e applicazioni del secondo
  principio. Campi di forza e stato meccanico.
  %%%%%%%%%%%%%%%%%%%%%%%%%%%%%%%%%%%%%%%%%%%%%%%%%%%%%%%%%%%%%%%%%%%%%%%%%%%%%
  21) Campi conservativi. Sistemi di riferimento non inerziali e forze
  apparenti.
  %%%%%%%%%%%%%%%%%%%%%%%%%%%%%%%%%%%%%%%%%%%%%%%%%%%%%%%%%%%%%%%%%%%%%%%%%%%%%
  22) Forza di Coriolis, forza centrifuga. Sistemi di punti materiali e
  grandezze del sistema. Forze interne ed esterne.
  %%%%%%%%%%%%%%%%%%%%%%%%%%%%%%%%%%%%%%%%%%%%%%%%%%%%%%%%%%%%%%%%%%%%%%%%%%%%%
  23) Variazione del tempo delle quantit\'a nei sistemi, prima equazione
  cardinale e conservazioni.
  %%%%%%%%%%%%%%%%%%%%%%%%%%%%%%%%%%%%%%%%%%%%%%%%%%%%%%%%%%%%%%%%%%%%%%%%%%%%%
  24) Sistema di riferimento del centro di massa. Teoremi di K\"onig.
  %%%%%%%%%%%%%%%%%%%%%%%%%%%%%%%%%%%%%%%%%%%%%%%%%%%%%%%%%%%%%%%%%%%%%%%%%%%%%
  25) Convenzione degli indici ripetuti. Delta di Kronecker. Simbolo di
  Levi-Civita (tensore completamente anti-simmetrico).
  %%%%%%%%%%%%%%%%%%%%%%%%%%%%%%%%%%%%%%%%%%%%%%%%%%%%%%%%%%%%%%%%%%%%%%%%%%%%%
  26) Relazione fra Delta di Kronecker e simbolo di Levi-Civita e applicazioni.
  %%%%%%%%%%%%%%%%%%%%%%%%%%%%%%%%%%%%%%%%%%%%%%%%%%%%%%%%%%%%%%%%%%%%%%%%%%%%%
  27) Definizioni: derivata parziale, differenziale di una funzione e
  gradiente. Teorema di Schwartz e matrice Hessiana.
  %%%%%%%%%%%%%%%%%%%%%%%%%%%%%%%%%%%%%%%%%%%%%%%%%%%%%%%%%%%%%%%%%%%%%%%%%%%%%
  28) Sviluppo in serie di Taylor in pi\'u variabili. Regola della catena.
  Matrice Jacobiana.
  %%%%%%%%%%%%%%%%%%%%%%%%%%%%%%%%%%%%%%%%%%%%%%%%%%%%%%%%%%%%%%%%%%%%%%%%%%%%%
  29) Trasformazioni controvarianti e covarianti. Derivata totale rispetto al
  tempo. Differenziale esatto.
  %%%%%%%%%%%%%%%%%%%%%%%%%%%%%%%%%%%%%%%%%%%%%%%%%%%%%%%%%%%%%%%%%%%%%%%%%%%%%
  30) Dipendenza temporale nella forza. Rotore, divergenza e laplaciano.
  %%%%%%%%%%%%%%%%%%%%%%%%%%%%%%%%%%%%%%%%%%%%%%%%%%%%%%%%%%%%%%%%%%%%%%%%%%%%%
  33) Percorso didattico per ricavare l'equazioni di Eulero-Lagrange.
  Impostazione del problema del pendolo semplice.
  %%%%%%%%%%%%%%%%%%%%%%%%%%%%%%%%%%%%%%%%%%%%%%%%%%%%%%%%%%%%%%%%%%%%%%%%%%%%%
  34) Soluzione diretta del pendolo semplice.
  %%%%%%%%%%%%%%%%%%%%%%%%%%%%%%%%%%%%%%%%%%%%%%%%%%%%%%%%%%%%%%%%%%%%%%%%%%%%%
  35) Indagine delle proiezioni nel sistema del pendolo semplice. Impostazione
  del problema del pendolo semplice trascinato.
  %%%%%%%%%%%%%%%%%%%%%%%%%%%%%%%%%%%%%%%%%%%%%%%%%%%%%%%%%%%%%%%%%%%%%%%%%%%%%
  36) Pendolo semplice trascinato: spostamenti virtuali, spostamenti reali e
  lavoro virtuale. Classificazione dei vincoli.
  %%%%%%%%%%%%%%%%%%%%%%%%%%%%%%%%%%%%%%%%%%%%%%%%%%%%%%%%%%%%%%%%%%%%%%%%%%%%%
  37) Vincoli nei pendoli. Vincoli: forma differenziale e integrabili. Sistemi
  a pi\'u vincoli e coordinate generalizzate.
  %%%%%%%%%%%%%%%%%%%%%%%%%%%%%%%%%%%%%%%%%%%%%%%%%%%%%%%%%%%%%%%%%%%%%%%%%%%%%
  38) Spazio tangente al vincolo. Variazione del vincolo rispetto alle
  coordinate generalizzate. Reazioni vincolari del vincolo liscio.
  %%%%%%%%%%%%%%%%%%%%%%%%%%%%%%%%%%%%%%%%%%%%%%%%%%%%%%%%%%%%%%%%%%%%%%%%%%%%%
  39) Impostazione del problema del pendolo doppio e equazioni del moto.
  %%%%%%%%%%%%%%%%%%%%%%%%%%%%%%%%%%%%%%%%%%%%%%%%%%%%%%%%%%%%%%%%%%%%%%%%%%%%%
  40) Soluzione del problema del pendolo doppio.
  %%%%%%%%%%%%%%%%%%%%%%%%%%%%%%%%%%%%%%%%%%%%%%%%%%%%%%%%%%%%%%%%%%%%%%%%%%%%%
  41) Principio di d'Alembert. Riscrittura dell'equazioni del moto in termini
  di coordinate generalizzate.
  %%%%%%%%%%%%%%%%%%%%%%%%%%%%%%%%%%%%%%%%%%%%%%%%%%%%%%%%%%%%%%%%%%%%%%%%%%%%%
  42) Spostamenti virtuali in termini di coordinate generalizzate. Forza
  generalizzata. Quantit\'a del moto in termini di energia cinetica.
  %%%%%%%%%%%%%%%%%%%%%%%%%%%%%%%%%%%%%%%%%%%%%%%%%%%%%%%%%%%%%%%%%%%%%%%%%%%%%
  43) Equazioni di (Eulero-)Lagrange. Momenti coniugati alle coordinate
  generalizzate. Indipendenza dalla scelta di coordinate generalizzate.
  %%%%%%%%%%%%%%%%%%%%%%%%%%%%%%%%%%%%%%%%%%%%%%%%%%%%%%%%%%%%%%%%%%%%%%%%%%%%%
  44) Vincoli come attrezzo algebrico per ricavare equazione generale
  indipendente. Cambio di coordinate generalizzate.
  %%%%%%%%%%%%%%%%%%%%%%%%%%%%%%%%%%%%%%%%%%%%%%%%%%%%%%%%%%%%%%%%%%%%%%%%%%%%%
  45) Risultato dell'ipotesi di cambio di coordinate invertibile.
  %%%%%%%%%%%%%%%%%%%%%%%%%%%%%%%%%%%%%%%%%%%%%%%%%%%%%%%%%%%%%%%%%%%%%%%%%%%%%
  47) Indagine sulle modifiche ammesse nella lagrangiana mantenendo inalterate
  l'equazioni di Eulero-Lagrange.
  %%%%%%%%%%%%%%%%%%%%%%%%%%%%%%%%%%%%%%%%%%%%%%%%%%%%%%%%%%%%%%%%%%%%%%%%%%%%%
  48) Invarianza dell'equazioni di Eulero-Lagrange rispetto a trasformazioni
  ristrette della lagrangiana.
  %%%%%%%%%%%%%%%%%%%%%%%%%%%%%%%%%%%%%%%%%%%%%%%%%%%%%%%%%%%%%%%%%%%%%%%%%%%%%
  49) Generalizzazione dell'invarianza per giusta scelta di lagrangiana.
  Relazione fra simmetrie e leggi di conservazione. Coordinate cicliche,
  integrali del moto. Esempio.
  %%%%%%%%%%%%%%%%%%%%%%%%%%%%%%%%%%%%%%%%%%%%%%%%%%%%%%%%%%%%%%%%%%%%%%%%%%%%%
  50) Ciclicit\'a dal giusto sistema di coordinate. Soluzione al problema di
  Keplero. Massimo numero di conservazioni.
  %%%%%%%%%%%%%%%%%%%%%%%%%%%%%%%%%%%%%%%%%%%%%%%%%%%%%%%%%%%%%%%%%%%%%%%%%%%%%
  80) Rappresentazioni del gruppo ortogonale su varie quantit\'a. E la
  rappresentazione fondamentale.
  %%%%%%%%%%%%%%%%%%%%%%%%%%%%%%%%%%%%%%%%%%%%%%%%%%%%%%%%%%%%%%%%%%%%%%%%%%%%%
  81) Rango del tensore e contrazione degli indici come operazione su di esso.
  Delta di Kronecker e simbolo di Levi-Civita come tensori invarianti.
  %%%%%%%%%%%%%%%%%%%%%%%%%%%%%%%%%%%%%%%%%%%%%%%%%%%%%%%%%%%%%%%%%%%%%%%%%%%%%
  82) 

\end{multicols}

\begin{multicols}{3}

  \mytheorem{Teorema di Noether}
  \footnote{
    \tiny Fonti:
    \url{https://en.wikipedia.org/wiki/Noether's_theorem},
    \url{https://it.wikipedia.org/wiki/Teorema_di_Noether},
    \url{https://proofwiki.org/wiki/Noether's_Theorem_(Calculus_of_Variations)}
  }
  \begin{equation}
    \label{th:noether}
    \begin{gathered}
      q_{\alpha} \to q'_{\alpha}
      \equiv q_{\alpha} + \varepsilon{A}
      \implies
      \dot{q}_{\alpha} \to \dot{q}'_{\alpha}
      = \dot{q}_{\alpha} + \varepsilon\dot{A} \\
      \dv{\varepsilon}
      L(q', \dot{q}', t) = 0
      \implies
      \dv{t}\left(\sum_\alpha p_{\alpha}A_{\alpha}\right) = 0
    \end{gathered}
  \end{equation}
  \myparagraph{Informalmente}
  If a system has a continuous symmetry property, then there are corresponding
  quantities whose values are conserved in time.
  \myparagraph{Formalmente}
  To every continuous symmetry generated by local actions there corresponds a
  conserved current and vice versa.

  \mytheorem{Identit\'a di Jacobi}
  \footnote{
    \tiny Fonti:
    \url{https://en.wikipedia.org/wiki/Jacobi_identity}
  }
  Siano $+$ e $\cross$ rispettivamente due operazioni binarie in cui $0$
  rappresenta l'identit\'a della prima, allora
  \begin{equation}
    \label{th:id-jacobi}
    x \cross (y \cross z) + y \cross (z \cross x) + z \cross (x \cross y) = 0
  \end{equation}
  (\'e semplice da ricordare perch\'e le variabili permutano ciclicamente)

  \mytheorem{Trasformata di Legendre}
  Data una funzione convessa.
  Vogliamo una funzione nuova costruita interamente dalla derivata della prima
  e che preserva tutta l'informazione.
  Una \textit{involuzione} \'e tipo $f(x) = -x$, se la applichi due volte torna
  al suo valore originale. Anche la trasformata di Legendre \'e un involuzione.
  Moltiplicare un numero complesso per $i$ \'e una 4-involuzione $f(z) = zi$.
  \begin{equation}
    \label{th:transform-legendre}
    f^*(\vec{v}) \equiv \vec{u}\cdot\vec{v} - f(\vec{u}),
    \quad \vec{v} \equiv \grad{\vec{u}}
  \end{equation}
  la propriet\'a ganza \'e che applicandola due volte ridiventa la funzione originale
  \begin{equation*}
    (f^*)^*(\vec{w}) = \vec{v}\cdot\vec{w} - f^*(\vec{v}) = f(\vec{u})
  \end{equation*}

  \mydefinition{Chiralit\'a}
  La chiralit\'a, dal greco $\chi\varepsilon i\rho$ (kh\'eir) "mano", \'e la
  propriet\'a di un oggetto rigido (o di una disposizione spaziale di punti o
  atomi) di essere non sovrapponibile alla sua immagine speculare.

  \mydefinition{Parentesi di Poisson}
  \begin{equation}
    \label{eq:poisson-brackets}
    \{f, g\} = \sum_i^N \left(
      \pdv{f}{q_i}\pdv{g}{p_i} - \pdv{f}{p_i}\pdv{g}{q_i}
    \right)
  \end{equation}
  Propriet\'a
  \begin{equation}
    \label{th:poisson-brackets}
    \begin{gathered}
      \{f, g\} = -\{g, f\} \\
      \{af + bg, h\} = a\{f, h\} + b\{g, h\} \\
      \{fg, h\} = \{f, h\}g + f\{g, h\} \\
      \{f, \{g, h\}\} + \{g, \{h, f\}\} + \{h, \{f, g\}\} = 0
    \end{gathered}
  \end{equation}
  

\end{multicols}

\end{document}
