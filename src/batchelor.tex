\documentclass[8pt, reqno, hidelinks]{amsart}
%\documentclass[8ptpt]{report}
\usepackage[margin=1.0cm]{geometry}
\usepackage[utf8]{inputenc}
\usepackage{multicol, wrapfig, blindtext, booktabs}
\usepackage{amsmath, amsthm, amsfonts, amssymb, mathrsfs, physics}
\usepackage{stackrel}
\usepackage{dsfont}
\usepackage{graphicx}
\usepackage{booktabs}
\usepackage{minted}
\usepackage{hyperref, bookmark}
\usepackage{cancel}
\usepackage{extarrows}
\usepackage{pagecolor}
\usepackage{tikz}
\usepackage{tikz-cd}
\usepackage{svg}
\usepackage{subcaption}
\usepackage{mathtools}
\usepackage{centernot}
\usepackage{xcolor}
\DeclarePairedDelimiter{\ceil}{\lceil}{\rceil}
\newenvironment{Figure}
  {\par\medskip\noindent\minipage{\linewidth}}
  {\endminipage\par\medskip}

% Standard Sets
\newcommand{\naturals}{\mathbb{N}}
\newcommand{\whole}{\mathbb{Z}}
\newcommand{\rationals}{\mathbb{Q}}
\newcommand{\reals}{\mathbb{R}}
\newcommand{\complex}{\mathbb{C}}
\newcommand{\field}{\mathbb{K}}
\newcommand{\cm}{\text{CM}}
\newcommand{\due}[1]{{^\star}#1}

\definecolor{blue}{HTML}{222255} %#0044cc
\definecolor{red}{HTML}{552222} %#ff3333
\definecolor{light-green}{HTML}{cdff99}
\definecolor{light-purple}{HTML}{ffcccc}
\definecolor{light-orange}{HTML}{ffeec0} %#ffdd99



%\newcommand{\mychapter}[1]{\section*{\texorpdfstring{\colorbox{light-purple}{#1}}{#1}}}
\newcommand{\mychapter}[1]{\section*{\texorpdfstring{\fbox{#1}}{#1}}}
\newcommand{\mysection}[1]{\section*{\texorpdfstring{\colorbox{light-orange}{#1}}{#1}}}
\newcommand{\mydefinition}[1]{\subsection*{\textcolor{red}{#1}}}
\newcommand{\mytheorem}[1]{\subsection*{\textcolor{blue}{#1}}}
\newcommand{\myparagraph}[1]{\paragraph{\textbf{#1}}}
\newcommand{\myproof}[1]{\tiny{\textcolor{gray}{#1}}}

\usepackage[english]{babel}

\begin{document}

\mychapter{Keywords}
\begin{multicols}{3}
  Material derivative
  Rate of expansion
  Vorticity
  Rate-of-strain tensor
  Scalar potential of an irrotational velocity distribution
  Vector potential of a solenoidal velocity distribution
  Stream function for a solenoidal velocity distribution
  Two-dimensional flow
  Axisymmetric flow
  Stress tensor
  Conservative body force per unit mass
  Inertia force
  Vortex-line
  Line vortex


  Solids, liquids and gases


  The continuum hypothesis
  Body forces (long range forces)
  Short range forces
  Local stress (force per unit area)


  Volume forces and surface forces acting on a fluid:
  1. Representation of surface forces by the stress tensor
  2. The stress tensor in a fluid at rest
  Stress tensor
  Normal stresses
  tangential stresses
  principal stresses
  stress tensor in a fluid at rest
  static-fluid pressure


  Mechanical equilibrium of a fluid:
  1. A body 'floating' in fluid at rest
  2. Fluid at rest under gravity
  Body fordce per unit volume
  Uniform density
  A body floating ina fluid at rest
  Archimedes theorem
  Fluid at rest under gravity

  Classical thermodynamics
  Parameters of state
  Equation of state
  Internal energy
  Adiabatic
  Specific heat
  Bulk modulus of elasticity
  Coefficient of compressibility
  Isentropic
  Enthalpy
  Helmholtz free energy
  Maxwell's thermodyanmic relations
  Coefficient of thermal expansion


  Transport phenomena:
  1. The linear relation between flux and the gradient of a scalar intensity
  2. The equations for diffusion and heat conduction in isotropic media at rest
  3. Molecular transport of momentum in a fluid
  Diffusion of matter
  Conduction of heat
  Simple shearing motion
  Internal friction
  Viscous fluid
  Linear relation between flux and the gradient of a scalar intensity
  flux vector
  Transport coefficient
  Homogeneous
  Isotropic
  Transported down the gradient
  Constitutive relations
  Equations for diffusion and heat conduction in isotropic media at rest
  Diffusion equation
  Coefficient of diffusion
  Coefficient of self-diffusion
  Thermal conductivity
  Thermal diffusivity
  Molecular transport of momentum in a fluid
  Viscosity
  Kinematic viscosity
  Diffusivities


  The distinctive properties of gases:
  1. A perfect gas in equilibrium
  2. Departures from the perfect-gas laws
  3. Transport coefficients in a perfect gas
  4. Other manifestations of departure from equilibrium of a perfect gas
  Perfect gas
  Perfect gas in equilibrium
  Boltzmann distribution
  Maxwell distribution of molecular velocities
  Principle of equipartition of energy
  Boltzmann's constant
  Equation of state for a perfect gas
  Carnot's law
  Perfect-gas with constant specific heats
  Departures from the perfect-gas law
  Van der Waal's equation
  Transport coefficients in a perfect gas
  Other manifestations of departure from equilibrium of a perfect gas
  Time of relaxation



  The distinctive properties of liquids:
  1. Equilibrium properties
  2. Transport coefficients
  Condensed phases
  Equilibrium properties
  Transport coefficients



  Conditions at a boundary between two media:
  1. Surface tension
  2. Equilibrium shape of a boundary between two stationary fluids
  3. Transition relations at a material boundary
  Surface tension
  Interface contraction
  Equilibrium shape of a boundary between two stationary fluids
  Pressure
  Capillarity
  Transition relations at a material boundary

  Specification of the flow field:
  eulerian specification
  lagrangian specification
  material volumes
  streamline
  stream-tube
  path
  streak line
  two-dimensional
  1. Differentiation following the motion of the fluid.
  local rate
  convective rate

  Conservation of mass:
  rate of expansion
  incompressible
  solenoidal
  1. Use of stream function to satisfy the mass-conservation equation.
  stream function
  Stokes stream function

  Analysis of the relative motion near a point:
  pure straining motin
  rigid-body rotation
  local vorticity
  irrotational curl
  1. Simple shearing motion.

  Expression for the velocity distribution with the specified rate of expansion and vorticity

  Singularities in the rate of expansion. Sources and sinks
  point source
  source doublet
  flux per uni lenght

  The vorticity distribution:
  vortex line
  vortex tube
  strenght of the vortex tube
  circulation
  1. Line vortices.
  vortex doublet
  2. Sheet vortices.
  sheet vortex

  Velocity distributions with zero rate of expansion and zero vorticity:
  irrotational solenoidal vector fields
  velocity potential
  Laplace's equation
  harmonic functions
  singly-connected
  single-valued function
  1. Conditions for $\nabla \psi$ to be determined uniquely.
  2. Irrotational solenoidal flow near a stagnation point.
  stagation point
  3. The complex potential for irrotational solenoidal flow in two dimensions.
  analytic function
  conjugate functions
  complex potential
  
  Irrotational solenoidal flow in doubly-connected regions of space:
  n-ply connected
  reconciliable region
  cyclic constant
  cyclic region
  acyclic region
  1. Conditions for $\nabla \psi$ to be determined uniquely.

  Three-dimensional flow fields extending to infinity
  1. Asymptotic expressions for $u_e$ and $u_v$.
  2. The behaviour of $psi$ at large distances.
  3. Conditions for $\nabla \psi$ to be determined uniquely.
  4. The expression of $psi$ as a power series.
  spherical solid harmonic
  5. Irrotational solenoidal flow due to a rigid body in translation motion.
  
  Two-dimensional flow fields extending to infinity:
  1. Irrotational solenoidal flow due to rigid body in translation motion.


\end{multicols}

\end{document}
