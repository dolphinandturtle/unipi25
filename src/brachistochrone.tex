\documentclass[8pt, reqno, hidelinks]{amsart}
%\documentclass[8ptpt]{report}
\usepackage[margin=1.0cm]{geometry}
\usepackage[utf8]{inputenc}
\usepackage{multicol, wrapfig, blindtext, booktabs}
\usepackage{amsmath, amsthm, amsfonts, amssymb, mathrsfs, physics}
\usepackage{stackrel}
\usepackage{dsfont}
\usepackage{graphicx}
\usepackage{booktabs}
\usepackage{minted}
\usepackage{hyperref, bookmark}
\usepackage{cancel}
\usepackage{extarrows}
\usepackage{pagecolor}
\usepackage{tikz}
\usepackage{tikz-cd}
\usepackage{svg}
\usepackage{subcaption}
\usepackage{mathtools}
\usepackage{centernot}
\usepackage{xcolor}
\DeclarePairedDelimiter{\ceil}{\lceil}{\rceil}
\newenvironment{Figure}
  {\par\medskip\noindent\minipage{\linewidth}}
  {\endminipage\par\medskip}

% Standard Sets
\newcommand{\naturals}{\mathbb{N}}
\newcommand{\whole}{\mathbb{Z}}
\newcommand{\rationals}{\mathbb{Q}}
\newcommand{\reals}{\mathbb{R}}
\newcommand{\complex}{\mathbb{C}}
\newcommand{\field}{\mathbb{K}}
\newcommand{\cm}{\text{CM}}
\newcommand{\due}[1]{{^\star}#1}

\definecolor{blue}{HTML}{222255} %#0044cc
\definecolor{red}{HTML}{552222} %#ff3333
\definecolor{light-green}{HTML}{cdff99}
\definecolor{light-purple}{HTML}{ffcccc}
\definecolor{light-orange}{HTML}{ffeec0} %#ffdd99



%\newcommand{\mychapter}[1]{\section*{\texorpdfstring{\colorbox{light-purple}{#1}}{#1}}}
\newcommand{\mychapter}[1]{\section*{\texorpdfstring{\fbox{#1}}{#1}}}
\newcommand{\mysection}[1]{\section*{\texorpdfstring{\colorbox{light-orange}{#1}}{#1}}}
\newcommand{\mydefinition}[1]{\subsection*{\textcolor{red}{#1}}}
\newcommand{\mytheorem}[1]{\subsection*{\textcolor{blue}{#1}}}
\newcommand{\myparagraph}[1]{\paragraph{\textbf{#1}}}
\newcommand{\myproof}[1]{\tiny{\textcolor{gray}{#1}}}

\usepackage[italian]{babel}


\begin{document}


\begin{multicols}{3}
  \mysection{Calcolo frazionario}
  \mydefinition{Funzione gamma}
  La \textit{funzione gamma} \'e un estensione nota della successione fattoriale dall'insieme dei numeri
  \textit{naturali} a quello dei \textit{reali positivi} (anche \textit{reali} e \textit{complessi}).
  A partire dalla definizione ricorsiva di fattoriale si potrebbe provare a risolvere tale espressione
  funzionale per determinarne la funzione generale per $x > 0$.
  \footnote{\url{https://it.wikipedia.org/wiki/Funzione_Gamma}}
  \begin{align}
    \label{eq:gamma-implicit}
    \begin{split}
      f(1) &= 1 \\
      f(x+1) &= xf(x)
    \end{split}
  \end{align}
  una funzione che soddisfa questa propriet\'a \'e la nota funzione gamma; lo si verifica direttamente
  \textit{integrando per parti}
  \begin{equation}
    \label{eq:gamma-function}
    \boldsymbol{\Gamma(x)} = \int_0^{+\infty}t^{x-1}e^{-t} \dd{t}
  \end{equation}
  \'e importante chiedersi per\'o se questa funzione \'e unica o se ne esistono altre, ma per mezzo del
  teorema di \textit{Bohr-Mollerup}
  \footnote{\url{https://en.wikipedia.org/wiki/Bohr-Mollerup_theorem}}
  se si aggiunge il vincolo aggiuntivo di ``super-convessit\'a''
  \begin{equation}
    \label{eq:super-convexity}
    y(x) = \ln f(x) \quad \text{\'e convessa}
  \end{equation}
  allora il teorema garantisce che la funzione \'e univocamente determinata, ovvero tutte le sue
  ``rappresentazioni'' sono equivalenti fra cui molte quella appena data da $\Gamma(x)$.

  \mydefinition{Integrale frazionario}
  La definizione di integrale frazionario \'e un estensione all'operatore d'integrale sensata se
  si \'e a conoscienza della \textit{formula di Cauchy per integrali ripetuti}
  \footnote{\url{https://en.wikipedia.org/wiki/Cauchy_formula_for_repeated_integration}}
  (si verifica direttamente per induzione)
  \begin{equation}
    \label{eq:cauchy-formula}
    I^n[f](x, x_0) = \frac{1}{(n-1)!} \int_{x_0}^x (x-t)^{n-1}f(t)\dd{t}
  \end{equation}
  la forma chiusa banalmente permette di evitare i problemi di unicit\'a che si incontrarono definendo
  la \textit{funzione gamma}, le funzioni che hanno ad argomento un numero naturale possono ora essere
  estese
  \begin{equation}
    \label{eq:cauchy-formula}
    I^n[f](x, x_0) = \frac{1}{\Gamma(n-1)} \int_{x_0}^x (x-t)^{n-1}f(t)\dd{t}
  \end{equation}

  \mydefinition{Derivata frazionaria}
  Rispetto all'estensione dell'integrale ripetuto, per il quale esisteva una forma chiusa che
  ``disaccoppiava'' il termine iterativo dall'operatore, la derivata ripetuta non presenta
  immediatamente una tale convenienza.
  
  \mysection{Problema della Brachistocrona}
  La legge di conservazione dell'energia in per un corpo un punto materiale in caduta libera pu\'o essere
  riscritta in forma infinitesimale abbusando la notazione dell'operatore di derivata
  \begin{equation}
    \begin{gathered}
      \frac{1}{2}m\left(\dv{l}{t}\right)^2 + mg(y-y_0) = 0 \\
      \text{d}t = \frac{1}{\sqrt{2g}}\frac{1}{\sqrt{y_0 - y}}\text{d}l
    \end{gathered}
  \end{equation}
  quest'ultima espressione pu\'o essere riscritta in forma integrale standard, ma lo si fa in termini della
  variabile $y$ perch\'e questo permetter\'a di esplicitare la sua forma frazionaria
  \begin{equation}
    \begin{gathered}
      T(y_0) = \frac{1}{\sqrt{2g}} \int_{0}^{y_0} \frac{1}{\sqrt{y_0 - y}}\dv{l}{y}\text{d}y \\
      \frac{T(y_0)}{\Gamma(1/2)} = \frac{1}{\sqrt{2g}}\frac{1}{\Gamma(1/2)}\int_{0}^{y_0} (y_0 - y)^{\frac{1}{2}-1}\dv{l}{y}\text{d}y \\
      T_0\sqrt{\frac{2g}{\pi}} = I^{\frac{1}{2}}\left(\dv{l}{y}\right)
    \end{gathered}
  \end{equation}
  quest'ultima espressione pu\'o essere ulteriormente semplificata utilizzando la derivata frazionaria,
  in questo caso molto semplice da calcolare dato il termine costante a sinistra.
  \begin{equation}
    \begin{gathered}
      D^{\frac{1}{2}}\left(T_0\sqrt{\frac{2g}{\pi}}\right) = D^{\frac{1}{2}}\left(I^{\frac{1}{2}}\left(\dv{l}{y}\right)\right) \\
      T_0\sqrt{\frac{2g}{\pi}}\frac{1}{\sqrt{y}} = \dv{l}{y} \\
    \end{gathered}
  \end{equation}
  a questo punto il problema \'e risolto e si ricava da semplici passaggi algebrici l'equazione della brachistocrona, una cicloide.
  \begin{equation}
    \dv{x}{y} = \sqrt{\frac{K^2}{y} - 1} ~, \quad K \equiv \frac{T_0\sqrt{2g}}{\pi}
  \end{equation}

\end{multicols}

\end{document}
