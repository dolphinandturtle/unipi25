\documentclass[8pt, reqno, hidelinks]{amsart}
%\documentclass[8ptpt]{report}
\usepackage[margin=1.0cm]{geometry}
\usepackage[utf8]{inputenc}
\usepackage{multicol, wrapfig, blindtext, booktabs}
\usepackage{amsmath, amsthm, amsfonts, amssymb, mathrsfs, physics}
\usepackage{stackrel}
\usepackage{dsfont}
\usepackage{graphicx}
\usepackage{booktabs}
\usepackage{minted}
\usepackage{hyperref, bookmark}
\usepackage{cancel}
\usepackage{extarrows}
\usepackage{pagecolor}
\usepackage{tikz}
\usepackage{tikz-cd}
\usepackage{svg}
\usepackage{subcaption}
\usepackage{mathtools}
\usepackage{centernot}
\usepackage{xcolor}
\DeclarePairedDelimiter{\ceil}{\lceil}{\rceil}
\newenvironment{Figure}
  {\par\medskip\noindent\minipage{\linewidth}}
  {\endminipage\par\medskip}

% Standard Sets
\newcommand{\naturals}{\mathbb{N}}
\newcommand{\whole}{\mathbb{Z}}
\newcommand{\rationals}{\mathbb{Q}}
\newcommand{\reals}{\mathbb{R}}
\newcommand{\complex}{\mathbb{C}}
\newcommand{\field}{\mathbb{K}}
\newcommand{\cm}{\text{CM}}
\newcommand{\due}[1]{{^\star}#1}

\definecolor{blue}{HTML}{0044cc}
\definecolor{red}{HTML}{ff3333}
\definecolor{light-green}{HTML}{cdff99}
\definecolor{light-purple}{HTML}{ffcccc}
\definecolor{light-orange}{HTML}{ffdd99}



%\newcommand{\mychapter}[1]{\section*{\texorpdfstring{\colorbox{light-purple}{#1}}{#1}}}
\newcommand{\mychapter}[1]{\section*{\texorpdfstring{\fbox{#1}}{#1}}}
\newcommand{\mysection}[1]{\section*{\texorpdfstring{\colorbox{light-orange}{#1}}{#1}}}
\newcommand{\mydefinition}[1]{\subsection*{\textcolor{red}{#1}}}
\newcommand{\mytheorem}[1]{\subsection*{\textcolor{blue}{#1}}}
\newcommand{\myparagraph}[1]{\paragraph{\textbf{#1}}}
\newcommand{\myproof}[1]{\tiny{\textcolor{gray}{#1}}}

\usepackage[english]{babel}


\title{``Note del corso di geometria differenziale'' by Diego Conti}

\begin{document}
\maketitle

\begin{equation*}
  \boxed{
  \text{Spazi topologici}
  \to \text{Variet\'a}
  \to \text{Spazio tangente}
  \to \text{Campo di vettori}
  \to \text{Metriche pseudo-Riemanniane}
  \to \text{Connessione di Levi-Civita}
  \to \text{Curvatura}
}
\end{equation*}


\mychapter{Keywords}

\begin{multicols}{3}
  Spazi metrici e topologici,
  metrica,
  metrica/distanza indotta,
  palle,
  intorni,
  aperti,
  sistema fondamentale di intorni,
  spazio topologico,
  topologia,
  chiusi,
  topologia di sottospazio,
  topologia discreta,
  topologia cofinita,
  base della topologia,
  spazio topologico di hausdorff,
  spazio proiettivo,
  topologia quoziente,
  funzioni continue,
  abbreviazione di funzioni,
  funzioni di proiezione,
  omeomorfismo,
  connessioni e connessione per archi,
  spazio topologico connesso,
  intervalli connessi,
  cammino in uno spazio topologico,
  spazio topologico connesso per archi,
  pidocchio sul pettine,
  spazio topologico localmente connesso per archi,
  ricoprimento di uno spazio topologico,
  spazio toplogico compatto,
  atlante differenziabile,
  carte,
  cambiamento di carte,
  struttura differenziabile,
  variet\'a differenziabile,
  proiezione stereografica,
  coordinate omogenee,
  funzioni $C^\infty$,
  diffeomorfismo,
  spazio tangente,
  spazio differenziabile,
  vettore tangente,
  differenziale,
  fibrazione di hopf,
  immersioni,
  sommersioni,
  sotto-variet\'a,
  punto regolare,
  valore regolare,
  embedding,
  fibrato tangente,
  derivazione,
  campi di vettori,
  sezione del fibrato tangente,
  campi di vettori locali,
  algebra di lie,
  prodotto di lie,
  f-correlato,
  orbita,
  curva integrale,
  aperto radiale,
  flusso di un campo di vettori,
  gruppi di lie,
  isomorfismo lineare,
  campo di vettori invariante a sinitra,
  sotto-gruppo,
  sotto-gruppo normale,
  sotto-gruppo analitico,
  quaternioni,
  prodotti tensoriali,
  algebra esterna,
  applicazione multi-lineare,
  bilineare,
  prodotto tensore,
  isomorfismo canonico,
  mappe multi-lineari alternanti,
  propriet\'a universale per mappe k-multilineari alternanti,
  p-q shuffle,
  fibrato dei tensori,
  fibrato delle forme differenziali,
  contrazione del tensore,
  k-forma differenziale,
  forma differenziale,
  algebra graduata,
  anti-derivazione,
  derivata esterna,
  pull-back,
  forma di volume,
  orientazione su una varit\'a,
  atlante orientato,
  orientazione standard,
  variet\'a con bordo,
  atlante con bordo,
  struttura differenziabile con bordo,
  spazio delle k-forme a supporto compatto,
  teorema di stokes generalizzato,
  metrica pseudo-riemanniana,
  metrica lorentziana,
  vettori di tipo luce,
  variet\'a pseudo-riemanniana,
  equivalenza conforme,
  tensore di weyl,
  inversione sferica,
  fibrati vettoriali,
  connessioni,
  derivata covariante,
  simboli di christoffel,
  forma di connessione,
  connessione di levi-civita,
  geodetiche,
  mappa esponenziale,
  aperto stellato,
  intorno normale,
  coordinate normali,
  palla geodetica,
  lunghezza,
  lemma di gauss,
  funzionale di energia,
  formula della variazione prima,
  variazione propria,
  teorema di hopf-rinow,
  spazio completo,
  completezza geodetica,
  tensore di riemann,
  curvatura di superfici,
  curvatura sezionale,
  curvature principali,
  curvatura gaussiana,
  tensore di weingarten,
  operatore di forma,
  equazione di gauss,
  curvatura di ricci,
  curvatura di weyl,
  seconda identit\'a di bianchi,
  teorema di schur,
  tensore di Ricci
\end{multicols}

\mychapter{Pages}
\begin{multicols}{3}
  1)
\end{multicols}

\mychapter{Propositions}
\begin{multicols*}{3}
  \mydefinition{Metric space}
  A set paired with a \textit{metric} binary real function acting over it such that:
  \begin{itemize}
  \item Only equal elements map to zero.
  \item The map is symmetric.
  \item The triangle inequality holds.
  \end{itemize}

  \mydefinition{Topological space}
  A set paired with a \textit{topology} containing its subsets relabled as \textit{open}, such that:
  \begin{itemize}
  \item The empty set and the whole set is open.
  \item Arbitrary unions of opens is open.
  \item Finite intersections of opens is open.
  \end{itemize}

  \mydefinition{Topological basis}
  A set of opens such that every element in the topology can be represented as sum of these.

  \mydefinition{Hausdorff space}
  A topological space that for each distinct pair of elements has a partitioned pair of opens that contains them.

  \mydefinition{Metric continuity}
  A function is metrically continuous in a center if every image's ball contains a ball's image.

  \mydefinition{Topological continuity}
  A function is topologically continuous in a point if every image's neighbour contains an neighbour's image.

  \mydefinition{Function abbreviation}
  Function restriction over a superset of its image contained in the codomain.

  \mydefinition{Omeomorphism}
  A continuous function with continuous inverse.
  
  \mydefinition{Connected topology}
  One of two equivalent conditions hold:
  \begin{itemize}
  \item Every clopen set is the space itself or the empty set.
  \item There are not two open partitions of the whole space.
  \end{itemize}

  \mydefinition{Walk}
  A continuous function from the unitary real interval to the space.

  \mydefinition{Connected by arches topology}
  For every two distinct points a walk can be made.

  \mydefinition{Localy connected by arches topology}
  For every point there is a neighbour to which a walk can be made.

  \mydefinition{Connected components}
  Idk.

  \mydefinition{Open covering}
  A subset of elements of the topology whose union is the whole space.

  \mydefinition{Compactness}
  A topological space is compact if every open covering has a finite subcovering.

  \mydefinition{Manifold}
  A Hausdorff space with a numerable base, a differential atlas over the space is given by
  a open covering and a set of omeomorphisms mapping to opens of $\reals^n$ such that for every
  ``cambiamento di carte'' the back and forth map is smooth.

\end{multicols*}

\end{document}
