\documentclass[8pt, reqno, hidelinks]{amsart}
%\documentclass[8ptpt]{report}
\usepackage[margin=1.0cm]{geometry}
\usepackage[utf8]{inputenc}
\usepackage{multicol, wrapfig, blindtext, booktabs}
\usepackage{amsmath, amsthm, amsfonts, amssymb, mathrsfs, physics}
\usepackage{stackrel}
\usepackage{dsfont}
\usepackage{graphicx}
\usepackage{booktabs}
\usepackage{minted}
\usepackage{hyperref, bookmark}
\usepackage{cancel}
\usepackage{extarrows}
\usepackage{pagecolor}
\usepackage{tikz}
\usepackage{tikz-cd}
\usepackage{svg}
\usepackage{subcaption}
\usepackage{mathtools}
\usepackage{centernot}
\usepackage{xcolor}
\DeclarePairedDelimiter{\ceil}{\lceil}{\rceil}
\newenvironment{Figure}
  {\par\medskip\noindent\minipage{\linewidth}}
  {\endminipage\par\medskip}

% Standard Sets
\newcommand{\naturals}{\mathbb{N}}
\newcommand{\whole}{\mathbb{Z}}
\newcommand{\rationals}{\mathbb{Q}}
\newcommand{\reals}{\mathbb{R}}
\newcommand{\complex}{\mathbb{C}}
\newcommand{\field}{\mathbb{K}}
\newcommand{\cm}{\text{CM}}
\newcommand{\due}[1]{{^\star}#1}

\definecolor{blue}{HTML}{222255} %#0044cc
\definecolor{red}{HTML}{552222} %#ff3333
\definecolor{light-green}{HTML}{cdff99}
\definecolor{light-purple}{HTML}{ffcccc}
\definecolor{light-orange}{HTML}{ffeec0} %#ffdd99



%\newcommand{\mychapter}[1]{\section*{\texorpdfstring{\colorbox{light-purple}{#1}}{#1}}}
\newcommand{\mychapter}[1]{\section*{\texorpdfstring{\fbox{#1}}{#1}}}
\newcommand{\mysection}[1]{\section*{\texorpdfstring{\colorbox{light-orange}{#1}}{#1}}}
\newcommand{\mydefinition}[1]{\subsection*{\textcolor{red}{#1}}}
\newcommand{\mytheorem}[1]{\subsection*{\textcolor{blue}{#1}}}
\newcommand{\myparagraph}[1]{\paragraph{\textbf{#1}}}
\newcommand{\myproof}[1]{\tiny{\textcolor{gray}{#1}}}


\begin{document}

\section{``Note del corso di geometria differenziale'' by Diego Conti}

\begin{equation*}
  \boxed{
  \text{Spazi topologici}
  \to \text{Variet\'a}
  \to \text{Spazio tangente}
  \to \text{Campo di vettori}
  \to \text{Metriche pseudo-Riemanniane}
  \to \text{Connessione di Levi-Civita}
  \to \text{Curvatura}
}
\end{equation*}
\\
Spazi metrici e topologici.
Metrica.
Metrica/distanza indotta.
Palle
Intorni
Aperti
Sistema fondamentale di intorni
Spazio topologico
Topologia
Chiusi
Topologia di sottospazio
Topologia discreta
Topologia cofinita
Base della topologia
Spazio topologico di Hausdorff
Spazio proiettivo
Topologia quoziente
Funzioni continue
Abbreviazione di funzioni
Funzioni di proiezione
Omeomorfismo
Connessioni e connessione per archi
Spazio topologico connesso
Intervalli connessi
Cammino in uno spazio topologico
Spazio topologico connesso per archi
Pidocchio sul pettine
Spazio topologico localmente connesso per archi
Ricoprimento di uno spazio topologico
Spazio toplogico compatto
Atlante differenziabile
Carte
Cambiamento di carte
Struttura differenziabile
Variet\'a differenziabile
Proiezione stereografica
Coordinate omogenee
Funzioni $C^\infty$
Diffeomorfismo
Spazio tangente
Spazio differenziabile
Vettore tangente
Differenziale
Fibrazione di Hopf
Immersioni
Sommersioni
Sotto-variet\'a
Punto regolare
Valore regolare
Embedding
Fibrato tangente
Derivazione
Campi di vettori
Sezione del fibrato tangente
Campi di vettori locali
Algebra di Lie
Prodotto di Lie
f-correlato
Orbita
Curva integrale
Aperto radiale
Flusso di un campo di vettori
Gruppi di Lie
Isomorfismo lineare
Campo di vettori invariante a sinitra
Sotto-gruppo
Sotto-gruppo normale
Sotto-gruppo analitico
Quaternioni
Prodotti tensoriali
Algebra esterna
Applicazione multi-lineare
Bilineare
Prodotto tensore
Isomorfismo canonico
Mappe multi-lineari alternanti
Propriet\'a universale per mappe k-multilineari alternanti
p-q shuffle
Fibrato dei tensori
Fibrato delle forme differenziali
Contrazione del tensore
K-forma differenziale
Forma differenziale
Algebra graduata
Anti-derivazione
Derivata esterna
Pull-back
Forma di volume
Orientazione su una varit\'a
Atlante orientato
Orientazione standard
Variet\'a con bordo
Atlante con bordo
Struttura differenziabile con bordo
Spazio delle k-forme a supporto compatto
Teorema di Stokes generalizzato
Metrica pseudo-riemanniana
Metrica lorentziana
Vettori di tipo luce
Variet\'a pseudo-riemanniana
Equivalenza conforme
Tensore di Weyl
Inversione sferica
Fibrati vettoriali
Connessioni
Derivata covariante
Simboli di Christoffel
Forma di connessione
Connessione di Levi-Civita
Geodetiche
Mappa esponenziale
Aperto stellato
Intorno normale
Coordinate normali
Palla geodetica
Lunghezza
Lemma di Gauss
Funzionale di energia
Formula della variazione prima
Variazione propria
Teorema di Hopf-Rinow
Spazio completo
Completezza geodetica
Tensore di Riemann
Curvatura di superfici
Curvatura sezionale
Curvature principali
Curvatura gaussiana
Tensore di Weingarten
Operatore di forma
Equazione di Gauss
Curvatura di Ricci
Curvatura di Weyl
Seconda identit\'a di Bianchi
Teorema di Schur
Tensore di Ricci

\end{document}
