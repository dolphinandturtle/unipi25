\documentclass[8pt, reqno, hidelinks]{amsart}
%\documentclass[8ptpt]{report}
\usepackage[margin=1.0cm]{geometry}
\usepackage[utf8]{inputenc}
\usepackage{multicol, wrapfig, blindtext, booktabs}
\usepackage{amsmath, amsthm, amsfonts, amssymb, mathrsfs, physics}
\usepackage{stackrel}
\usepackage{dsfont}
\usepackage{graphicx}
\usepackage{booktabs}
\usepackage{minted}
\usepackage{hyperref, bookmark}
\usepackage{cancel}
\usepackage{extarrows}
\usepackage{pagecolor}
\usepackage{tikz}
\usepackage{tikz-cd}
\usepackage{svg}
\usepackage{subcaption}
\usepackage{mathtools}
\usepackage{centernot}
\usepackage{xcolor}
\DeclarePairedDelimiter{\ceil}{\lceil}{\rceil}
\newenvironment{Figure}
  {\par\medskip\noindent\minipage{\linewidth}}
  {\endminipage\par\medskip}

% Standard Sets
\newcommand{\naturals}{\mathbb{N}}
\newcommand{\whole}{\mathbb{Z}}
\newcommand{\rationals}{\mathbb{Q}}
\newcommand{\reals}{\mathbb{R}}
\newcommand{\complex}{\mathbb{C}}
\newcommand{\field}{\mathbb{K}}
\newcommand{\cm}{\text{CM}}
\newcommand{\due}[1]{{^\star}#1}

\definecolor{blue}{HTML}{0044cc}
\definecolor{red}{HTML}{ff3333}
\definecolor{light-green}{HTML}{cdff99}
\definecolor{light-purple}{HTML}{ffcccc}
\definecolor{light-orange}{HTML}{ffdd99}



%\newcommand{\mychapter}[1]{\section*{\texorpdfstring{\colorbox{light-purple}{#1}}{#1}}}
\newcommand{\mychapter}[1]{\section*{\texorpdfstring{\fbox{#1}}{#1}}}
\newcommand{\mysection}[1]{\section*{\texorpdfstring{\colorbox{light-orange}{#1}}{#1}}}
\newcommand{\mydefinition}[1]{\subsection*{\textcolor{red}{#1}}}
\newcommand{\mytheorem}[1]{\subsection*{\textcolor{blue}{#1}}}
\newcommand{\myparagraph}[1]{\paragraph{\textbf{#1}}}
\newcommand{\myproof}[1]{\tiny{\textcolor{gray}{#1}}}

\usepackage[english]{babel}

\begin{document}

\mychapter{Keywords}
\begin{multicols}{3}
  Set operations:
  member, inclusion, union, intersection, difference.
  Map properties:
  Injective/into, surjective/onto, bijection/one-to-one.
  Equivalence relation, equivalence properties:
  reflexivity, symmetry and transitivity.
  Group axioms, non-commutative (or abelian) groups, discrete groups,
  infinite groups, continuous groups, order of the group.
  Group examples:
  additive group $\whole$,
  additive group $\reals$,
  multiplicative group $\reals^+$,
  permutations group $S_N$,
  general linear group $GL(N, \complex)$ or $GL(N, \reals)$,
  special linear group $SL(N, \complex)$ or $SL(N, \reals)$,
  unitary group $U(N)$,
  special unitary group $SU(N)$,
  orthogonal group $O(N)$,
  Symplectic group $Sp(2N, \reals)$ or $Sp(2N, \complex)$,
  euclidean group $E_n$,
  Lorentz group $SO(3, 1)$,
  Poincar\'e group.

  Ring axioms
  Division ring
  Field as commutative division ring
  Continuous division rings

  Subgroup condition
  Invariant subgroup
  Normal subgroup
  Simple subgroup
  Semi-simple subgroup
  Group homomorphism
  Kernel of homomorphism
  Group isomorphism
  Group automorphism
  Group inner automorphism
  Left coset

  Representation
  Equivalente representation
  Unitary representation

  Commutator
  Center element of the group
  Conjugacy class
  Cyclic group
  Dihedral group

  Lie group axioms
  Invariant integration
  Haar measure
  Group compactness

  Lie algebras
  Structure constants
  Exponential parametrization
  Jacobi identity
  Adjoint representation
  Invariant sub-algebra / ideal
  Simple sub-algebra
  Semi-simple sub-algebra
  Algebra center
  Cartan criterion
  Killing form
  Metric tensor
  Casimir operators

  Covering map
  Sheets
  Universal covering space

  Conncted group as a manifold

  Homotopy groups
  Fundamental group / First homotopy group

  Monodromy groups

  Projective spaces

  Higher homotopy groups

  Hopf map

  Representation theory
  Representation space
  Equivalent representations
  Direct-product representation

  (Ir)reducible representations

  Unitary representations

  (Pseudo-)Real/Complex representations

  Shur's first lemma
  Shur's second lemma

  Character of a representation
  Completeness of a representation

  Isospin

  Fundamental and adjoint representation
  Construction of irreducible representations
  Clebsh-Gordan coefficients

  Quark model
  Young tableaux
  Weight vectors

  From the Quark model to the contemporary theory of fundamental
  interactions

  Asymptotic freedom

  Quark confinement

  Coset construction

  Induced representations

  Spinor representations
  Chiralities

  Zero momentum states

  Timelike momenta

  Lightlike momenta

  Root vectors
  Simple root vectors
  Recostruction from simple roots
  Weight vectors

  Dynkin diagrams
\end{multicols}


\begin{multicols}{3}
  %pag 54
  $$N(g^{-1}) = N^{-1}(g)$$
  $$N_{lk}(g)^* = N_{kl}(g)^{\dagger} = N_{kl}(g^{-1})$$


  Esempio:
  \begin{equation}
    \begin{gathered}
      SO(2) \sim U(1) \\
      M(g) = e^{im\phi}, \quad m = 0, \pm 1, \pm 2, ... \\
      \int\text{d}g = \int \frac{\text{d}\phi}{2\pi} e^{im\phi}e^{-in\phi} = 
      \begin{cases}
        0 & \text{if}~m\neq n\\
        1 & \text{if}~m=n
      \end{cases}
    \end{gathered}
  \end{equation}

  Esempio di completezza rappresentazione generica:
  \begin{equation}
    \begin{gathered}
      SO(2) \sim U(1) \\
      M(g) = e^{im\phi} \\
      \sum_{m=-\infty}^{+\infty} e^{im\phi} e^{-im\phi'} = \delta(\phi-\phi')
    \end{gathered}
  \end{equation}

  \begin{align*}
    SU(2)& \to \text{isospin} \\
    \begin{matrix}
      SO(2) \\
      \sim U(1)
    \end{matrix}
    &\to \text{quantum field theory} \\
    SU(3)& \to \text{chromodinamica}
  \end{align*}

  Fisica nucleare:
  \begin{enumerate}
  \item $V_{pp} = V_{nn}\quad$ simmetria di carica
  \item $V_{pp} = V_{nn} = V_{np}\quad$ invarianza di carica
  \item $
    \begin{bmatrix}
      p \\
      n
    \end{bmatrix}
    \to U
    \begin{bmatrix}
      p \\
      n
    \end{bmatrix}
    \quad
    $ Potenziale di Yukawa
  \end{enumerate}

  Isotopi:
  $^{14}C \quad ^{14}N \quad ^{14}O$


\end{multicols}

\end{document}
